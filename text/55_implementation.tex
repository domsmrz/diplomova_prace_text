\chapter{Implementation}

The goal of this chapter is to summarize the architecture of the underlying implementation and general design. For the detailed guide on how to use it for \gls{iden} generation on a new data please read \autoref{ch:guide}. For the implementation details, please refer to attached code.

As we mentioned in the \autoref{sec:vl}, whole our work is designed as a module for \gls{VL}. Even though our work is largely independent of other modules within the framework of \gls{VL}, it shares one main component and that is the relation database. We use the database as an \gls{api} for communication for other modules as well as a place to store our intermediate results.

In our implementation we used and tested usage of PostgreSQL as an management system. However, as we use SQLAlchemy library (\cite{sqlalchemy}) for all the communication with the database, so usage of another management system should be without any problems.

Overall, the implementation is split into a few parts. Each part processes the input from the database and writes there also the results. The primary purpose of each part is to be used as a python module. However, for the purposes of testing and simple evaluation, we also add an command line interface for each part.

One of the parts is responsible for creating feature vectors described in \autoref{ch:features}. For the histogram approaches the process is straight-forward. The \glspl{det} from the selected cameras are querried, the histograms computed and then commited back to the database. For the approaches involving annotation with neural network we can supply the network externally, in such case the process is similar. We also have an option to train the network on the data in the database. In such case the training data are first retrieved, then the network is trained and stored locally for subsequent usages. After the network is trained the data for annotation is annotated using the network and the feature vectors are again committed into the database.

The other part is responsible for clustering the \glspl{det} (\autoref{ch:iden_construction}), that is for both -- trajectory generation and identity generation. For this process the \glspl{det} form given cameras are again retrieved from the database along, if applicable, with the corresponding feature vectors created with the selected settings. If we use the constriction of trajectories from before they are also queried. Once the \glspl{det} are clustered according to the parameters, the resulting trajectories or \glspl{iden} are again commited to the database for processing via other \gls{VL} modules.

We also provide a tool for manual annotation of the \glspl{det} that we used for obtaining the ground truth used for the experiments. The tool queries the \glspl{det} from the database as well as selected \glspl{iden} or trajectory. User then can see the the \glspl{det} frame-by-frame as well as their division to \glspl{iden}. It is possible to re-assign the wrongly assigned \glspl{det} to different \gls{iden}. This approach allows for semi-automated anno