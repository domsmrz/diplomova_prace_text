\chapter{User Guide}

\label{ch:guide}

In this chapter we review all the steps needed to use the attached software to process new \glspl{det}.

\section{Installation}

In the attached code you can find a dirctory \texttt{videolytics} containing a single subdirectory \texttt{ReID}. We do not include other parts of \gls{VL} projects (which would populate the root directory) as they are not part of our work. Nevertheless, we keep the structure of the project in order not to break path for imports.

Whole application is written in Python only. We developed it and tested it with Python 3.8.5 on Linux KDE Neon 20.04. However, we do not expect compatibility issues with earlier version of Python or different operating systems.

For the full functionality several Python packages are needed. They are listed in \texttt{requirements.txt}. You can install them via:

\begin{verbatim}
    pip install -r ReID/requirements.txt
\end{verbatim}

Note that within the requirements there is \verb+psycopg2-binary+ used for communication with PostgreSQL database. If you are using different management system, other packages may be needed. However due to abstraction layer provided by SQLAlchemy no changes in the code should be needed. Also note that you want to use GPU for the \glspl{nn} you will need to install CUDA.\footnote{\url{https://docs.nvidia.com/cuda/index.html}} Please refer to the Nvidia website on how to install it.

The last step to get working environment is to setup a database connection. You need to edit the file \texttt{ReID/database\_settings}. In the file an URL of the database needs to be assigned to the variable \verb+database+. The URL should be of following form:

\begin{verbatim}
    dialect[+driver]://user:password@host/dbname[?key=value..]
\end{verbatim}

The database should have required tables as discussed in \autoref{ch:implementation}. If the tables are not present, they could be created via:

\begin{verbatim}
    python -m ReID.database_connection
\end{verbatim}

However, keep in mind that the software does not process raw videos, but needs already prepared \glspl{det}. That is the tables \texttt{detection}, \texttt{camera} and \texttt{frame} should be populated.

\section{Feature vectors}

In order to annotate the \glspl{det} you need to run following command:

\begin{verbatim}
    python -m ReID.annotator
\end{verbatim}

The annotator has several global options such as which camera to annotate (in terms of the \verb+id+ in the database). For the full list of options run the command with \verb+-h+ flag. The last command of the annotator should be the type of annotation to use. Several annotators require additional parameters, in such case add the parameters after the annotator. You can also use the \verb+-h+ flag after the annotator to list all the available options. For example to annotate camera 42 by RGB histograms with four bins per channal and no background filtering you would run:

\begin{verbatim}
    python -m ReID.annotator -c 42 simple-rgb-histogram -b 4
\end{verbatim}

In case of the approaches based on the \glspl{nn} the parameters also specify the training dataset. If the network with the exactly same parameters is already trained from the previous runs, the training is skipped and the trained network is used.

\section{Detection Clustering}

For the clustering the \glspl{det}, that is for creation of trajectories and \gls{iden} we use script \texttt{ReID/trajectory\_generator}. Similarly to the annotator we can use the \verb+-h+ option to list available flags. As we cluster the detection based on the generated feature vectors we also need to supply the annotator with corresponding options. If the annotation with corresponding options are already present in the database, the feature vectors are queried directly. Otherwise the feature vectors are firtly computed from the detections and commited to the database. An example of command for generating \glspl{iden} in camera 42 based on direct approach is:

\begin{verbatim}
    python -m ReID.trajectory_generator -c 42 -r \
        direct --iou 0.8 -c 0.3 -t 0.2 simple-rgb-histogram -b 5
\end{verbatim}


\section{Annotation Tool}

To invoke annotation tool we use \texttt{ReID/explore} script. We also need to supply \verb+id+ of trajectory or identity model we wish to explore, or a file with a clustering. We can show \glspl{det} from camera 42 with identity model 196,884 we can run:

\begin{verbatim}
    python -m ReID.explore -c 42 -i 196884
\end{verbatim}

To list additional parameters please use \verb+-h+ flag.

\subsection{User Interface}

After launching the application, the \glspl{det} from the first frame of the corresponding camera (or cameras) are displayed. Example of such screen is shown in figure \ref{fig:annotation_tool}.

After launching the application, we can see highlighted detections from
the first frame of the video on the screen.
Only detections are shown, not the whole video, in this way, we do not need access
to the original video (just data already presented in the database). In the top left corner
of the screen, we can see the sequence number of the frame (this is not the frame ID
in the corresponding database, just a sequence number within the respective stream). Alongside
the sequence number is displayed the timestamp of the displayed frame.

We can browse the video by right and left arrows, skipping one frame forward and
backward, respectively. Each time we skip on a different frame, the detections from
the new frame are rendered and highlighted. However, the old detections are left
on the screen for better referencing the previous screen. The user can refresh the display
and hide all detections apart from detections from the current frame by hitting \verb+c+.
For other commands allowing to browse the stream refer to table
\ref{tab:annotation_basic_commands}

By clicking on the detection with a mouse button user can select given detection (or rather trajectory/identity of the corresponding detection). One identity can be selected
by the left mouse button and one other by the right mouse button. Selected identities can then be unified via \verb+unify+ command. One can also select a detection with the
middle mouse button. Unlike selection with the left and right mouse button, the application
allows for multiple identities to be selected with the middle mouse button. Such selection
does not serve the purpose of annotation directly (does not affect \verb+unify+ command).
However, it marks the given identity as already processed (see ``Up'' and ``Down'' arrows in \autoref{tab:annotation_basic_commands}).

\begin{table}
    \centering
    \begin{tabularx}{\textwidth}{l|X}
         \textbf{Key} & \textbf{Command} \\ \hline
         Right arrow & Skip one frame forward \\ \hline
         Left arrow & Skip one frame back \\ \hline
         Up arrow & Skip frames until the one with detection that is not selected
         via left, right or middle mouse button or to the end of the stream \\ \hline
         Down arrow & Skip frames backwards until the one with detection that is not selected via left, right or middle mouse button or to the first frame \\ \hline
         Page up & Skip 10 frames forward \\ \hline
         Page down & Skip 10 frames back \\ \hline
         Home & Skip to the first frame \\ \hline
         End & Skip to the last frame
    \end{tabularx}
    \caption{Basic controls of annnotation tool}
    \label{tab:annotation_basic_commands}
\end{table}

Each selection is associated with a different highlight color for given detections.
For explanation of given colors, please refer to table \ref{tab:annotation_highlight}

\begin{table}[]
    \centering
    \begin{tabularx}{\textwidth}{l|X}
         \textbf{Color} & \textbf{Explanation} \\ \hline
         Blue & Unselected detections that are assigned into an identity \\ \hline
         Yellow & Detections that are present on the frame but are not part of any
         identity for loaded model \\ \hline
         Red & Detections/identities selected via left mouse button \\ \hline
         Green & Detections/identities selected via right mouse button \\ \hline
         Cyan & Detections/identities highlighted via middle mouse buttn \\
    \end{tabularx}
    \caption{Highlight color of detections}
    \label{tab:annotation_highlight}
\end{table}

\subsection{Fast exploration of interesting detections}

\label{subsec:exploration}

Aside from the standard controlling mechanism, we offer additional features especially
useful for exploring incorrectly identified and annotated identities. We allow
for a loading list of pairs of identities during the exploration of the results. These
identities can be quickly searched through. By this method, we an, for example, efficiently fix the incorrectly annotated data in the golden dataset if we obtain a list of ``suspicious'' identities (for example list of identities that our re-identification algorithm classified differently than are classified in the golden dataset).

The list itself should state two identities ID (corresponding to the IDs in the
database) on each line, separated by space. The lines that start with the hash (\verb+#+) are considered as comments and not processed. Once the pairs are loaded
from the file with \verb+load_pairs+ command, given pairs can be processed via other
commands, refer to section \ref{sec:commands} for a complete list of commands.

\subsection{Commands}
\label{sec:commands}

By typing colon (\verb+:+) the user enters the command mode. Then the user can type
a command which will be displayed at the lower part of the screen. Commands are confirmed
by hitting enter and canceled via Escape. List of available command is:

\begin{itemize}
    \item \verb+unify+ (or hitting \verb+u+ in standard mode) -- merge all detections
    from identity selected by right mouse button into identity selected by left mouse 
    button
    \item \verb+hide+ (or hitting \verb+h+ in standard mode) -- hide the identity
    currently selected with the left mouse button (detections from given identity
    will be treated as if they would not be part of any identity at all, i.e.
    highlighted in yellow if all the detections are visible, or not displayed at all
    if user hid them via \verb+all+ command).
    \item \verb+all+ (or hitting \verb+a+ in standard mode) -- stops showing detections
    that are not part of any identity. If these detections are already hidden, start
    showing them instead
    \item \verb+quit+ (or hitting \verb+q+ or \verb+Esc+ twice in standard mode) -- terminates
    the application
    \item \verb+left+ (or hitting \verb+l+ in standard mode) -- find and skip to the
    first frame where is detection selected by left mouse button. Frames are searched
    starting from the current frame onwards (previsou frames are not searched at all)
    \item \verb+right+ (or hitting \verb+r+ in standard mode) -- works same as the 
    \verb+left+ command, except it searches for detections selected with right mouse
    button
    \item \verb+transfer+ (or hitting \verb+t+ in standard mode) -- move the detection
    from the identity selected by right mouse button that is currently displayed from
    its identity into identity selected by left mouse button.
    \item \verb+single_auto+ (or hitting \verb+s+ in standard mode) -- create new
    cluster and assign currently displayed detection that is part of the
    cluster selected by right mouse button into it.
    \item \verb+id+ (or hitting \verb+i+ in standard mode) -- start showing IDs of
    detections below the currenly highlighted detections. These correspond to the IDs
    in the \texttt{detection} table.
    \item \verb+memory+ (or hitting \verb+m+ in standard mode) -- repeat the last search 
    for deteciton (invoked with commands
    like \verb+left+, \verb+right+, \verb+find+ or \verb+next+
    \item \verb+postpone+ (or hitting \verb+p+ in standard mode) -- mark last searched 
    pair of identities as not resolved. This pair will appear in form of comment should
    the pairs be saved in this session.
    \item \verb+next+ (or hitting \verb+n+ in standard mode) -- search for next pair
    of identities loaded via \verb+load_pairs+ command. This will automatically skip
    to the first frame with the detection from the first identity. Furthermore, the
    first identity will be selected with left mouse button and the second with the
    right mouse button. Given pair will be marked as processed, meaning that it will
    not appear in the file for subsequent \verb+save_pairs+ command (unless the pair
    is postponed via \verb+postpone+ command).
    \item \verb+clear+ (or hitting \verb+c+ in standard mode) -- refresh the view.
    This will delete all the detections displayed for other frames and leaves only
    the detections from current frame.
    \item \verb+delete+ (or hitting \verb+d+ in standard mode) -- remove identity
    selected by left mouse button from the moden entirely
    \item \verb+save[!] <file_path>+ -- store the current model into a local file
    specificied via \verb+<file_path>+ if given file does not exists. To override the file if it exists add \verb+!+
    % \item \verb+save! <file_path>+ -- works same as the \verb+save+ command except
    % it overrites given file if it exists
    \item \verb+upload <generator_name> <options>+ -- saves current clustering into the
    database as a model with given parameters
    \item \verb+find <detection_id> [(l|r|<detection_id>)]+ -- skips to a frame
    with detection with given \verb+<detection_id>+. If the ID is followed by \verb+l+,
    automatically select the detection via left mouse button, if it is followed by
    \verb+r+ select it via right mouse button. If the ID is followed by second
    \verb+<detection_id>+, select the detection corresponding to the first ID by left
    mouse button, and the detection corresponding to the second ID by right mouse button
    \item \verb+single <detection_id>+ -- create a new identity/trajectory. Assign the
    detection with \verb+<detection_id>+ into it.
    \item \verb+load_pairs <file_path>+ -- load list of pairs of identities (as per subsection \ref{subsec:exploration}). 
    \item \verb+save_pairs[!] <file_path>+ -- save all remaining pairs of identities that were loaded via \verb+load_pairs+ and not processed via \verb+next+ command. The
    pairs are saved to the corresponding \verb+<file_path>+, unless given file already
    exists. To override the file if it exists add \verb+!+
    % \item \verb+save_pairs[!] <file_path>+ -- works the same as the \verb+save_pairs+
    % command, except it overrides the file \verb+<file_path>+ if it exists
    \item \verb+select_left <identity_id>+ -- select identity \verb+<identity_id>+ 
    as with the left mouse button and
    skip to the first frame with the detection from given identity
    \item \verb+select_right <identity_id>+ -- works the same as \verb+select_left+
    command, except it select given identity with the right mouse button.
    \item \verb+skip <n>+ -- skips \verb+n+ frames forward (a negative number can be supplied to skip backwards)
\end{itemize}

