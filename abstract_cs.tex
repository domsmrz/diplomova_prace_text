%%% Šablona pro jednoduchý soubor formátu PDF/A, jako treba samostatný abstrakt práce.

\documentclass[12pt]{report}

\usepackage[a4paper, hmargin=1in, vmargin=1in]{geometry}
\usepackage[a-2u]{pdfx}
\usepackage[czech]{babel}
\usepackage[utf8]{inputenc}
\usepackage[T1]{fontenc}
\usepackage{lmodern}
\usepackage{textcomp}

\begin{document}

%% Do not forget to edit abstract_cs.xmpdata.

Široké využití bezpečnostních kamer poskytuje velké množství dat, které lze dále využít v oblastech jako je bezpečnost a rozvoj města. Důležitým odrazovým můstkem pro extrakci užitečných informací je hledání stejného objektu v různých okamžicích nebo na různých kamerách. V této práci se zaměřujeme konkrétně na tuto část zpracování videa, obvykle označovanou jako re-identifikace.

Práce je rozdělena na dvě části. V první části se zaměřujeme na informace ohledně umístění detekovaného objektu v čase a prostoru. Ve druhé části kombinujeme tyto metadata s vizuální informací. Pro extrakci užitečných deskriptorů z obrázků používáme metody založené na distribuci barev i na metodách hlubokého učení. Abychom mohli vyhodnotit navrhované přístupy připravili jsme vlastní sadu detekcí. Poskytujeme i nástroj, který jsme k anotaci těchto dat použili.
\end{document}
