%%% The main file. It contains definitions of basic parameters and includes all other parts.

%% Settings for single-side (simplex) printing
% Margins: left 40mm, right 25mm, top and bottom 25mm
% (but beware, LaTeX adds 1in implicitly)
% \documentclass[12pt,a4paper]{report}
% \setlength\textwidth{145mm}
% \setlength\textheight{247mm}
% \setlength\oddsidemargin{15mm}
% \setlength\evensidemargin{15mm}
% \setlength\topmargin{0mm}
% \setlength\headsep{0mm}
% \setlength\headheight{0mm}
% % \openright makes the following text appear on a right-hand page
% \let\openright=\clearpage

%% Settings for two-sided (duplex) printing
\documentclass[12pt,a4paper,twoside,openright]{report}
\setlength\textwidth{145mm}
\setlength\textheight{247mm}
\setlength\oddsidemargin{14.2mm}
\setlength\evensidemargin{0mm}
\setlength\topmargin{0mm}
\setlength\headsep{0mm}
\setlength\headheight{0mm}
\let\openright=\cleardoublepage

%% Generate PDF/A-2u
\usepackage[a-2u]{pdfx}

%% Character encoding: usually latin2, cp1250 or utf8:
\usepackage[utf8]{inputenc}

%% Prefer Latin Modern fonts
\usepackage{lmodern}

%% Further useful packages (included in most LaTeX distributions)
\usepackage{amsmath}        % extensions for typesetting of math
\usepackage{amsfonts}       % math fonts
\usepackage{amsthm}         % theorems, definitions, etc.
\usepackage{amssymb}
\usepackage{bbding}         % various symbols (squares, asterisks, scissors, ...)
\usepackage{bm}             % boldface symbols (\bm)
\usepackage{graphicx}       % embedding of pictures
\usepackage{fancyvrb}       % improved verbatim environment
\usepackage{natbib}         % citation style AUTHOR (YEAR), or AUTHOR [NUMBER]
\usepackage[nottoc]{tocbibind} % makes sure that bibliography and the lists
			    % of figures/tables are included in the table
			    % of contents
\usepackage{dcolumn}        % improved alignment of table columns
\usepackage{booktabs}       % improved horizontal lines in tables
\usepackage{paralist}       % improved enumerate and itemize
\usepackage{xcolor}         % typesetting in color
\usepackage{tabularx}
\usepackage[toc,nogroupskip,nopostdot]{glossaries}
\usepackage{todonotes}
\usepackage{enumitem}

\usepackage{caption}
\usepackage{subcaption}
\usepackage{siunitx}

\usepackage{footnotehyper} % TODO


\setlist[enumerate]{topsep=0pt,itemsep=-1ex,partopsep=1ex,parsep=1ex}
%\usepackage{svg}

\usepackage{xcolor} % TODO TODO TODO
\usepackage{gensymb}
\usepackage{mathtools}
\usepackage{caption}
\usepackage[ruled,vlined]{algorithm2e}

\makenoidxglossaries
\loadglsentries{glossary}
%
\newglossaryentry{VL}
{
    name=Videolytics,
    description={Encapsulating system this work is module of}
}

\newglossaryentry{det}
{
    name=detection,
    %text=detection,
    description={image of a detected object within a frame with spatial metadata},
    first=\emph{detection},
    firstplural=\emph{detections},
}

\newglossaryentry{iden}
{
    name=identity,
    %text=identity,
    plural=identities,
    description={set of detections displaying the same object}
    first=\emph{identity},
    firstplural=\emph{identities},
}

\newglossaryentry{ses}
{
    name=session,
    %text=session,
    description={pair of set of detections and its associated ``golden'' partitioning},
    first=\emph{session},
    firstplural=\emph{sessions},
}

\newacronym{ml}{ML}{machine learning}
\newacronym{api}{API}{application programming interface}
\newacronym{mle}{MLE}{maximum likelihood estimator}
\newacronym{mse}{MSE}{mean square error}

\newglossaryentry{nn}
{
    name={neural network},
    description={popular machine learning model},
    plural={neural networks},
    first=\emph{neural network},
    firstplural=\emph{neural networks},
}

%%% Basic information on the thesis

% Thesis title in English (exactly as in the formal assignment)
\def\ThesisTitle{Re-identification of Objects in Video Stream using Data Analytics}

% Author of the thesis
\def\ThesisAuthor{Dominik Smrž}

% Year when the thesis is submitted
\def\YearSubmitted{2021}

% Name of the department or institute, where the work was officially assigned
% (according to the Organizational Structure of MFF UK in English,
% or a full name of a department outside MFF)
\def\Department{Department of Software Engineering}

% Is it a department (katedra), or an institute (ústav)?
\def\DeptType{Department}

% Thesis supervisor: name, surname and titles
\def\Supervisor{prof. RNDr. Tomáš Skopal, Ph.D.}

% Supervisor's department (again according to Organizational structure of MFF)
\def\SupervisorsDepartment{Department of Software Engineering}

% Study programme and specialization
\def\StudyProgramme{Computer Science}
\def\StudyBranch{Artificial Intelligence}

% An optional dedication: you can thank whomever you wish (your supervisor,
% consultant, a person who lent the software, etc.)
\def\Dedication{%
I would like to thank my supervisor, prof. RNDr. Tomáš Skopal, Ph.D. for all the patient guidance and valuable advice. I also want to extend my gratitude to my friends and family for all the support. Finally, I would like to thank my girlfriend for the enormous help and encouragement during my studies and especially while writing this thesis.
}

% Abstract (recommended length around 80-200 words; this is not a copy of your thesis assignment!)
\def\Abstract{%
The wide usage of surveillance cameras provides data that can be used in various areas, such as security and urban planning. An important stepping stone for useful information extraction is matching the seen object across different points in time or different cameras. In this work, we focus specifically on this part of the video processing, usually referred to as re-identification.

We split our work into two stages. In the first part, we focus on the spatial and temporal information regarding the detected objects. In the second part, we combine this metadata with the visual information. For the extraction of useful descriptors from the images, we use methods based on the color distribution as well as state-of-the-art deep neural networks. We also annotate a dataset to evaluate our approaches and provide a comprehensive evaluation. Additionally, we provide a custom tool we used to annotate the dataset.
}

% 3 to 5 keywords (recommended), each enclosed in curly braces
\def\Keywords{%
{Video Surveillance} {Deep Learning} {Person Re-identification}
}

%% The hyperref package for clickable links in PDF and also for storing
%% metadata to PDF (including the table of contents).
%% Most settings are pre-set by the pdfx package.
\hypersetup{hidelinks}
\hypersetup{unicode}
\hypersetup{breaklinks=true}

% Definitions of macros (see description inside)
\include{macros}

% Title page and various mandatory informational pages
\begin{document}
\include{title}

%%% A page with automatically generated table of contents of the master thesis

\tableofcontents

%%% Each chapter is kept in a separate file
\chapter*{Introduction}
\addcontentsline{toc}{chapter}{Introduction}

%123456789 123456789 123456789 123456789 123456789 123456789 123456789 123456789
Increasing usage of surveillance cameras brought the need for automatic object
re-identification.  The goal of the re-identification is to match the same
object (usually a person) across multiple cameras, or across different frames
of a single camera. This problem has many different application not only
directly in security but also in areas such as traffic monitoring and robotics.
Despite the extensive advancement in computer vision and artificial
intelligence re-identification still remains open problem.

Our task is to create a module for \gls{VL} -- an analytical system
for video surveillance. The purpose of this module is to perform
re-identification task on the videos within the system.

Aside from the actual re-identification algorithm, we present an entire
framework encapsulating the issue. We implement an easy-to-use pipleline, that
can be used for testing other approaches to the problem. We also implement
simple, yet effective, tool for annotation, that can be also used for
in-depth analysis of any given re-identification algorithm. Finally, we also
present the set of other utilities, used for evaluation of various
re-identification algorithms and other related tasks.

\todo[inline]{todo}
\chapter{Preliminaries}

\label{ch:preliminaries}

In this chapter, we provide an overview of the elementary concepts we use in the thesis. We start by introducing the basic notation we use thorough out the thesis. Then we show the concept of distances and metric spaces. We finish the chapter by describing the basics of neural networks and related concepts.

\section{Notation}

We use standard notation for number systems: $\mathbb{N}$ for natural numbers (including zero), $\mathbb{R}$ for real numbers, $\mathbb{R}^+$ for positive real numbers, and $\mathbb{R}^+_0$ for non-negative real numbers.

% We use $\Pt{X}$ to denote a power set, that is a set of all subsets of set $X$.

We often work with vectors, and we shall write them as $\vec{a}$. In order to refer to $i$-th element of a vector $\vec{a}$ we shall write $a_i$. Similarly, in the case of matrix $A$, we shall refer to a specific element in $i$-th row and $j$-th column as $A_{i,j}$.

Finally, we shall use the term \emph{tensor} as a generalization of vectors and matrices. We use the following recursive definition:

\begin{defn}
0-dimensional tensor is a single number. n-dimensional ($n > 0$) vector of sizes
$(s_1, s_2, \ldots, s_n)$ is an ordered set of $s_1$ $(n-1)$-dimensional
tensors of sizes $(s_2, s_3, \ldots, s_n)$.
\end{defn}

As we can see, a 1-dimensional tensor is a vector, and a 2-dimensional tensor is a matrix. Similarly, we shall refer to specific element of tensor $a$ using subscripts ($a_{i,j,k}$).

We also use some elementary operation on the sets. We denote the power set of $A$ by $\Pt{A}$, that is set of all subsets of $A$. We also use a term partitioning:

\begin{defn}
We say that set $B$ is partitioning of set $A$ if:
\begin{enumerate}[label=(\roman*)]
    \item $\bigcup B = A$
    \item $(\forall b \in B) (\forall b' \in B) (b \cap b' \neq \emptyset \Leftrightarrow b = b')$
\end{enumerate}
We refer to elements of $B$ as partitions.
\end{defn}

\section{Distance and Metric Spaces}
\label{sec:distances}

In this work, we often need to express the similarity between two data points (usually images or their representations). To provide a precise mathematical background for such similarity, we use the concept of metric spaces introduced by \cite{metric} and named by \cite{metricname}. We further differentiate between metrics (and metric spaces) and distances (and distance spaces). We use definitions from \cite{deza2009encyclopedia}.

\begin{defn}
Metric space $(\mathcal{F}, \delta)$ is a pair of set $\mathcal{F}$ and
corresponding distance function
$\delta : \mathcal{F} \times \mathcal{F} \goto \mathbb{R}_0^+$ such that
following axioms hold:
\begin{itemize}
    \item $\delta(x, y) = 0 \Leftrightarrow x = y$ (Coincidence axiom)
    \item $\delta(x, y) = \delta(y, x)$ (Axiom of symmetry)
    \item $\delta(x, y) \leq \delta(x, z) + \delta(z, y)$ (Triangle inequality)
\end{itemize}
\end{defn}%

However, as these axioms prove to be too restrictive in some cases, we also use the concept of distance space:
\begin{defn}
Distance space $(\mathcal{F}, \delta)$ is a pair of set $\mathcal{F}$ and
corresponding distance function $\delta : \mathcal{F} \times \mathcal{F} \goto \mathbb{R}_0^+$ such that
following axioms hold:
\begin{itemize}
    \item $\delta(x, y) = \delta(y, x)$
    \item $\delta(x, x) = 0$
\end{itemize}
\end{defn}%
It is easy to see that each metric space is also a distance space.

\subsection{Used Distances and Metrics}

\label{ssec:used_distances}

In this subsection, we briefly introduce the specific metrics and distances we use in this thesis.

\begin{defn}
Euclidean space is a metric space ($\mathbb{R}^n, \delta$) where the
distance function $\delta$ (referred as euclidean distance) is:
$$\delta(\vec{x}, \vec{y}) = \sqrt{\sum_{i=1}^n(x_i - y_i)^2}$$
\end{defn}

\begin{defn}
Manhattan space (more often referred as space L1) is a metric
space ($\mathbb{R}^n, \delta$) where the
distance function $\delta$ (referred as Manhattan distance) is:
$$\delta(\vec{x}, \vec{y}) = \sum_{i=1}^n \abs{x_i - y_i}$$
\end{defn}

Aside from the two introduced metric functions we also use one distance that does not fulfill the axioms for metric space. To properly define it, let us first introduce cosine similarity:

\begin{defn}
Cosine similarity is a function $S_C : \R^n \times \R^n \goto [-1, 1]$, s.t.:
$$S_C(\vec{x}, \vec{y}) = \mathrm{cos}(\vec{x}, \vec{y}) = \frac{\vec{x}\vec{y}}{||\vec{x}||\,||\vec{y}||}$$
\end{defn}
As the values of cosine similarity fall within the interval $[-1,1]$ we can
easily derive following cosine space:
\begin{defn}
Cosine distance space is a distance space ($\mathbb{R}^n, \delta$) where the distance
function $\delta$ (referred as cosine distance) is:
$$\delta(\vec{x}, \vec{y}) = 1 - S_C(\vec{x}, \vec{y})$$
where $S_C$ is cosine similarity.
\end{defn}

\section{Neural Networks}
\label{sec:nn}

One of the most popular \gls{ml} models in recent years are artificial \glspl{nn}. The general idea of a mathematical model inspired by a biological brain's function is nothing new (\cite{first_nn}). The discovery of an effective learning algorithm (\cite{backprop}) somewhat established the ``standard'' structure of so-called feed-forward \glspl{nn}.

These feed-forward \glspl{nn} can be viewed as a sequence of layers, where each layer represents a function $f^{(i)}: \R^p \goto \R^q$. The whole \gls{nn} is then a composition of all the layers in order, i.e., $f^{(n)}(f^{(n-1)}(\ldots f^{(1)}(\vec{x})\ldots))$. A function associated with each layer is most commonly a linear combination of the input values with offset (bias) which is then transformed using non-linear differentiable activation function (such as rectified linear unit abbreviated as ReLU\footnote{$\mathrm{ReLU}(x) = x$ for $x > 0$, otherwise $\mathrm{ReLU}(x) = 0$}). For example the $j$-th output of $i$-th layer could be computed as follows (where $\vec{\alpha}^{(i,j)}$ are parameters of the layer):

\begin{equation}
    \label{eq:dense}
    f^{(i)}_j(\vec{x}) = \mathrm{ReLU}\left(\alpha_{\mathrm{bias}}^{(i,j)} + \left(\vec{\alpha}^{(i,j)}\right)^T \vec{x}\right) = \mathrm{ReLU}\left(\alpha_{\mathrm{bias}}^{(i,j)} + \sum_{k}\alpha_k^{(i,j)}x_k\right)
\end{equation}

While this description of \glspl{nn} is somewhat simplified in comparison with the recurrent and residual connections in modern models, it provides reasonable idea of how the \glspl{nn} operate.

As in the case of other \glsentrylong{ml} models, the ultimate goal is to approximate some function $f^*$. While the function $f^*$ itself is usually unknown, there is a set of values (dataset) for which the desired output is know (i.e., pairs of $\vec{x}, \vec{y}$ where $f^*(\vec{x}) = \vec{y}$). How well the \gls{nn} approximates the function $f^*$ is then measured using the loss function. One of the most common loss function in the case where the output of the $f^*$ is a single number is \gls{mse}:

$$\text{MSE}_{f^*}(f) = \frac{1}{|X|} \sum_{\vec{x} \in X} (f(\vec{x}) - f^*(\vec{x}))^2$$

In order to use the \glspl{nn} to solve a given task, an architecture (that is number of layers, activation functions, number of neurons in each layer and similar) of the \gls{nn} is firstly appropriately chosen.\footnote{There is quite extensive research in algorithms to learn the appropriate architecture automatically, based on the presented dataset (for example \cite{neat}). However, such approaches are out-of-scope of this thesis.} Choosing the appropriate architecture for a given type of task is an open problem and target of numerous research articles and, to a degree, also this thesis. Once the architecture is chosen, the parameters (e.g., coefficients $\alpha^{(i,j)}$ in \autoref{eq:dense}) are \emph{learned} from the training dataset.

For training purposes, the \gls{nn} is perceived as a function of its parameters rather than the function of the input space. These parameters are then iteratively adjusted to minimize the chosen loss function with respect to the known (training) samples. The gradient descent algorithm (or similar) is usually used to find the minimum of the loss function. The general architecture of \glspl{nn} (linear combination and differentiable activation functions) allows for efficient computation of derivative of loss function w.r.t. the parameters of \gls{nn}. Therefore, it is possible to learn suitable parameters even for \glspl{nn} with thousands of parameters.

Recently one may also encounter expressions such as \emph{deep learning} or \emph{deep neural networks}. These terms also refer to the \glspl{nn} and, as the name suggests, they refer specifically to the \glspl{nn} with several layers. However, there is no specific number of layers when a \gls{nn} became deep. These terms are rather associated with various recently discovered techniques that allow effective learning and usage of multiple (even several hundred) layers.

In this section, we have reviewed the basic concepts of \glspl{nn}. We shall further proceed to elaborate on a few specifics that are relevant to the thesis. However, if the reader wishes to learn more about the general concept of the (deep) neural networks, we can recommend \cite{deeplearningbook} as an excellent starting point.

\subsection{Convolutional layers}
\label{sec:conv}

One specific type of \gls{nn} layer is a convolutional layer. These convolutional layers are popular in image processing \glspl{nn} and majority of \glspl{nn} in this area use them. They were successfully used by \cite{conv}. \cite{convbackprop} showed that corresponding coefficients could be learned via standard \gls{nn} learning algorithm.

Convolutional layers allow simple extraction of the same information from various parts of the input data while maintaining spatial information. They achieve this by using discrete convolution.\footnote{There are some differences compared to standard notation of convolution. Namely we sum over finite subset rather than all natural numbers, however to get the standard form we can pad the kernel with zeros outside of our finite subset. The second difference is that we iterate over $i+j$ rather than $i-j$, to obtain the original expression we need to just use ``mirrored'' kernel ($c'_i = c_{-i}$). As the kernels are initialized randomly and then learned, it does not matter if we use ``mirrored'' kernel or not.} In a simple case of input being just a vector, the corresponding function can be expressed as:
$$y_i = \sum_{j=1}^N x_{i+j} c_j$$

Where $\vec{x}$ and $\vec{y}$ are input and output vectors respectively and $\vec{c}$ is \emph{kernel}, i.e., vector of learned coefficients of size $N$.


However, the primary usage of these layers is not processing one-dimensional vectors. We often use them for processing images. Each image is by itself two-dimensional structure. Furthermore, we usually represent each element of the input matrix by several numbers, which means we have to process three-dimensional tensors. In the input layer information along the third axis usually corresponds to three color channels. In the layers further in the network, the information is more abstract. With the addition of this third dimension applying the convolutional layer becomes more complex, but the basic idea remains the same. For example the value of $r$-th channel in $(i, j)$-th position can be computed as follows ($s$ in the sum represents an index of the input channel):

$$y_{i, j, r} = \sum_{k, l, s} x_{i+k, j+l, s} \, c_{k, l, r, s}$$

One of the main advantages of the convolutional layers is that they have relatively low number of parameters. The number of parameters depends only on the size of the kernel and not the input size (aside from the number of channels). For example even for large inputs the small kernels (e.g. $3 \times 3$) is often used. For further description and motivation, please refer to the original work or \cite{deeplearningbook}.

\subsection{Transfer Learning}

\label{ssec:transfer_learning}

In this thesis, we shall heavily use a concept of so called transfer learning. This concept tries to leverage information gained on solving a different but related problem. In the context of \glspl{nn}, it mostly means that we re-use the architecture or part of it and learned weights of a \gls{nn} designed for a slightly different task.

% The advantages of using transfer learning on \glspl{nn} were first explored in 1976 and later revisited by \cite{transferreviewed}. However, since then many more research was based on this premise (for example, \cite{transferlearning}).

The advantages of using transfer learning on \glspl{nn} were first explored in 1976. Since then, many more research was based on this premise (for example, \cite{transferlearning}). The original article from 1976 was later revisited by its author (\cite{transferreviewed}).

Specifically in our case we shall re-use complex deep learning architectures (described in \autoref{sec:existing_architectures}) trained on ImageNet tasks (\cite{imagenetresults}). As these are classification tasks, the last layer is responsible for selecting the correct class. The idea is to remove this layer under the assumption that the previous layers capture the general information about the input image. This altered network can then be used directly as-is or fine-tuned by training on the samples from our task.

There are several advantages to using transfer learning. For one, it may significantly reduce the time required to learn suitable weights for our tasks. This is important as some of the original networks were trained for a considerable amount of time on powerful hardware. This allows us to use these models while spending just limited resources on fine-tuning the parameters. As the training on different tasks initiates the weights, it also circumvents the problems with proper initialization of the weights. Finally, we directly use architectures that were empirically proven to be suitable for the image processing task.

\section{Triplet Loss}

\label{sec:triplet_loss}

Given our task's character, standard loss function such as \gls{mse} described in \autoref{sec:nn} is not well suited for us. Nor we can use other often used loss function such as cross-entropy. Such loss function is designed for tasks where there are specific known labels for each given data point from the training set. However, in our task, we want to assign the images of people to clusters displaying the same person. Note that this is still a supervised task -- i.e., we know beforehand which images belong together, only in our case, there is no specific label assigned to a cluster.

Our goal is to achieve this clustering, which would only create separated clusters and always one cluster per person. To achieve the desired clustering, we need to penalize the ``within-cluster'' distance and reward the ``in-between cluster'' distance. In terms of the loss function, it means that the value of the loss function should increase when the distance between the two samples from the same cluster increases (or more precisely when the distance between the feature vectors corresponding to the same cluster increases). To maximize the ``in-between'' distance, the loss function should decrease as the distance between the samples from the different clusters increases. A visual representation of this process can be seen in the \autoref{fig:triplet_loss}.

\begin{figure}
    \centering
    \includegraphics[width=\textwidth]{img/triplet_loss.png}
    \caption[Triplet loss]{The Triplet loss minimizes the distance between an anchor and a positive and maximizes the distance between the anchor and the negative datapoint.\\
    Source: \cite{tripletlossnn}}
    \label{fig:triplet_loss}
\end{figure}

This is the basic idea behind the triplet loss introduced by \cite{tripletlossfirst} and then further formalized by \cite{tripletlosssecond} and subsequently used as a loss function for \glspl{nn} by \cite{tripletlossnn}.

We shall use the triplet loss in following form:
\begin{defn}
Triplet loss for selected triplet (A, P, N) is:
\begin{equation}
\mathcal{L}(A, P, N) = \text{max}(\delta(f(A), f(P)) - \delta(f(A), f(N)) + \alpha, 0)
\label{eq:triplet}
\end{equation}
Where $f$ is a transformation given by \gls{nn}, $\delta$
is a selected distance function and $\alpha$ is a priori selected margin. The
selected triplet is a triplet of specific samples -- $A$ is a generic point
called ``anchor'' point, $P$ is a ``positive'' sample (i.e., sample of the same
cluster as anchor point) and $N$ is a ``negative'' sample (i.e., sample of
different cluster than anchor point).
\end{defn}


Such a definition provides the formula of the loss for a specific triplet. However, we can easily extend this notion by considering the average loss over all selected triplets.

It can be easily seen that the minimal value of the Triplet Loss is zero. In order to achieve this optimal value, not only do all ``in-between'' cluster distances have to be greater than ``within'' cluster distances, but they must be greater by at least a given margin $\alpha$.

\subsection{Triplet Selection}

An additional challenge for using Triplet Loss is to select suitable triplets. Usually, each point in the dataset (or more precisely in a subset of the dataset called a batch) is once selected as an anchor point. Then the task is to find the appropriate positive and negative sample for each anchor point.

The related terminology is somewhat ambiguous. For practical purposes, we choose to use the terminology corresponding to the implementation in the Tensorflow framework (\cite{tensorflow}). They operate with two strategies: Hard and Semi-Hard Triplet Loss.

In the case of Hard Loss, the positive and negative samples are selected ``as hard as possible'', which means that the positive sample is the furthest (from the anchor point) out of all considered samples of the same cluster. On the other hand, the negative sample is the closest sample out of all samples from samples that are not of the same cluster as the anchor point.

However, this intuitive selection of triplets can lead to bad local minima (\cite{tripletlossnn}). Therefore we also evaluate a different approach for triplet selection. The Semi-Hard approach selects such a pair of positive and negative samples that the negative sample is further away from the anchor point than the positive one, but not by the required margin. In terms of \autoref{eq:triplet} it means that $0 < \delta(f(A), f(N)) - \delta(f(A), f(P)) < \alpha$.
\chapter{Related Work}

%123456789 123456789 123456789 123456789 123456789 123456789 123456789 123456789

\todo[inline]{Pridat marekove detekcie do related work}

Great advantages offered by usable re-identification algorithm brings also
considerable amount of the research in the area. In this chapter we aim to
briefly review this research and related papers. Furthermore we also focus on
the popular architecture of \glspl{nn} for image processing (not necessarily
used for re-identification), as we make use of them in this work.


\section{Person Re-identification}

% {\color{red}
% In order to re-identify the same person across time or different views various
% techniques has been employed. In most of the cases the goal of the research
% was to establish if the two images of a person displays the same person
% rather than to produce whole complete cluster of images corresponding to one
% person.
% }

% {\color{blue}
% Person re-identification aims to correctly match or discover images of the
% same person. These images can origin from different views or even from the
% same view, but happening at a different time.

% Most of the research in this area focuses on correctly predicting if the
% two provided images belong to the same person. This reduces the problem 
% to simple supervised task of with a clear yes/no answer. An alternative 
% approach would be rather clustering based on the person. It is clear,
% that from the full information of the first task we can construct the full
% results of the second one. The reason to think about the second option is XXXXX
% }

The problem of re-identification is an active research area for the past
few decades. Many different approach were explored by this date. Most
of the designs embed the input image of the person into a specific
feature space. Resulting feature vectors are then simply compared using
some standard distance functions. In a combination with a suitably selected
threshold this produces the desired answer whether there is a same person
on selected images (\cite{cheng2016person}) or not. However, even designs that seek to
directly answer the question without embedding the images into some intermediate feature
space were explored (\cite{li2014deepreid}).

\subsection{Early Research}
\label{sec:early_research}

% {\color{blue} 
% One of the first break-through in the person re-identification can be traced
% back to the \cite{krumm2000multi}. \todo{tak jak byl prvni?} This paper laid down a method using color
% histogram, which became well-known and widely used later on.

% In this histogram approach, the image is represented through its distribution 
% of the colors. Such obtained histograms are then compared between different
% images of the people to compare how much they match in the color distribution.

% The idea of this topic is still relevant. For example, authors of XXX (https://ieeexplore.ieee.org/document/7025495) aim to improve the histograms
% by incorporating also the spatial information. This histogram
% method is also often incorporated with more complex machine learning methods
% as \gls{em} or set of Gaussians (\todo[inline]{zdroje do zatvoriek, ale konkretne tie, ktore to pouzivaju na tuto ulohu}).
% }

% {\color{red}
% The earlier attempts to tackle this tasks incorporated color histograms
% (\cite{krumm2000multi}). Where the distribution of color on a person's
% image is aggregated into histogram. This histograms are then compared and
% based on their similarity the conclusion is drawn whether the displayed
% person is the same or not. This extraction if often accompanied by more
% complex techniques such as \gls{em} over a set of gaussians
% (\cite{orwell1999multi}).
% }

Many of the early attempts in tracking and re-identification describe the images
using colors and color histograms in particular. For example work presented by
\cite{krumm2000multi} shows usage of color histograms to perform tracking in
indoor environment.

\todo[inline]{tento paper krum2000multi by som realne rozpisala viac, lebo to je ten zaklad od coho sa clovek odrazi. Akoze co na tu dobu dosiahol a na co sa pozrel}

Another interesting research was shown by \cite{orwell1999multi}. Again the
authors extract color histograms. These histograms are then matched to identities
by using \gls{em} over a set of Gaussians.

Even though the use of color histograms was later somewhat overshadowed by
more complex techniques such as deep learning, some researches still use it
in order to perform re-identification. For example \cite{zeng2014person}
combines the color histogram with spatial information to obtain better results.

Additionally to the color histograms, we can see a variety of new techniques emerging. A technique
based on the principal axes is presented by \cite{hu2006principal}. Ability
to correctly identify the axes heavily relies on successful filtering out the
background. The filtering is especially challenging in the scenarios
where the occlusion is present or if people form are in a crowded area. 
The authors propose a special version of the approach to tackle these challenging scenarios.
Once the silhouette of a person is established, the principal axis is found.
Then the algorithm infers the position of the ``ground-point'', i.e. point
where the person touches the ground (this is usually the point where principal
axis intersects lower bound of the bounding box, but may not be so in case
of occlusion). Based on these ``ground-points`` and with use of Kalman filter
the algorithm then tracks the people. In order to re-identify the person
across multiple cameras, the homography is firstly computed using suitable
landmarks. Based on this obtained homography, it is straightforward to 
compute the sets of points in the view of each camera corresponding 
to the same ``ground-point''. To establish a suitable matching (re-identification)
of detected peoples, simple algorithm for maximum likelihood is used.

\todo[inline]{Tu by mohol prist este obrazok z toho hu2006principal s obrazkom dvoch kamier, lebo pises viac o homografii}

\subsection{Era of the Deep Neural Networks}

{\color{red}
In the past few years the models using deep learning approach started to
emerge. For example \cite{ding2015deep} introduced relatively simple
\gls{nn} consisting of a few convolutional and max-pooling layers,
with the final dense layer. They trained the network essentialy using
Triplet Loss (albeit using non-standard terminology). They also focused
on the suitable selection of triplets.
}

{\color{blue}
In the past few years, more and more successful models based on the
deep learning mode emerge. As many other image processing related tasks,
also this field came to a new era of proposed approaches based on the 
neural networks.

One of such research is presented by \cite{ding2015deep}. The authors propose
a relatively simple \gls{nn} consisting only of a few concolutional layers, 
intertwined with the max-pooling layers. To train the network, they use a Triplet 
loss (``albeit`` using non-standard terminology). They elaborate more on the
selection of the triplets to train on.
\todo[inline]{chyba tu realne povedat aky vysledok dosiahli, nech uvedies citatela}
\todo[inline]{chyba to u vsetkych a bolo by to vhodne}
}

A similar approach was explored by \cite{cheng2016person}. The basis
of their method is again convolutional \gls{nn} trained by Triplet Loss.
However, they presented several notable improvements. Firstly, the architecture
they designed after the first convolutional layer ``splits'' the input into
four horizontal strips. Meaning that each of these strips is fed into its
own independent part of the \gls{nn}, where each part tries to learn the
features of its respective body part. However, the original information as
a whole is also used (``coppied'') into fifth ``global'' part of the \gls{nn}
and trained. At the end of the \gls{nn} pipe-line, the outputs of the parts
is combined and the loss used is again the Triplet Loss. The authors also focused on the
formulation of the Triplet Loss. They propose slightly different formulation
of the Triplet Loss which is designed specifically for the re-identification
task.
\todo[inline]{obrazok toho modelu, tej pipeline}


Another \gls{nn} design is presented by \cite{li2014deepreid}. Differently from
other approaches, they do not use Triplet Loss or a similar loss to embed the original images into feature
space (which is then compared via euclidean distance). They design directly
filter pairing \gls{nn}, which takes two images as an input and combines them
after a first convolution and max-pooling layer. The output of the network is
then just a number encapsulating the ``confidence'' -- i.e. how likely is that the images
display the same person. The whole network is then trained via simple softmax
loss.

Additional work that tries to solve this problem with usage of deep \gls{nn}
is presented by \cite{hermans2017defense}. They experiment with the
hand-crafted architecture of the neural network, as well as the pretrained
ResNet model\todo{odkaz na resnet kap}. They further elaborate on the topic of Triplet Loss and 
suitable triplet selection.

Quite elaborate deep network architecture is offered by \cite{xu2018attention}.
Their Attention-Aware Compositional Network can be broken into several parts.
Firstly, the network tries to estimate attention map and a visibility score
for each body part (given body parts are a priory setup by researchers). Based
on the visibility, the features are extracted for the given parts. \todo{tu je myslienkovy skok z toho, ze maju jednotlive casti tela, do spojenia v base network. Neotruzmiem tomu uplne} Finally, the
feature vector representing the original image is produced by selected base
network.

\section{Image Processing Architectures}

\label{sec:existing_architectures}

In this section we further review few very popular \gls{nn} architectures
for the image processing. Even though they are not specifically designed for the
task of re-identification, we utilize both their architectures and the 
weights trained for classification tasks which we use via means of
the transfer learning\todo{odkaz}.

We present more in detail two well-known \gls{nn} -- MobileNet and ResNet.
Both of them presented a break-through in their own category at the time they were introduced.

\subsection{Residual Networks}

\label{ssec:resnet}

Residual networks (commonly referred as just ResNets) is a quite popular family
of \gls{nn} architectures designed especially for processing images. The first ResNet was
officially introduced and described by \cite{resnet} and later improved by
\cite{resnetimp}. This design gained much popularity by proving its qualities in many different tasks.
The ResNet became a state-of-art, widely known model. The model won the first places at ILSVRC competition on the
tasks of ImageNet detection, ImageNet localization (\cite{imagenetresults}) as
well as COCO detection and COCO segmentation at COCO competition
(\cite{cocodataset}).

Next, we further investigate the proposed ResNet architecture and the
motivations behind it. The authors tried to solve the problem of low training
accuracy of the deep neural networks compared to their shallower counter parts.
This counter-intuitive phenomenon led them to following design change.

For a given layer, instead of learning direct suitable mapping $H(\vec{x})$,
they designed the layer to learn just the ``residuum''
$H(\vec{x}) - \vec{x} = F(\vec{x})$ (hence the name). This was achieved by
adding residual connection between layer $n$ and layer $m > n$.
The output of layer $n$ and layer $m - 1$ would be added
together effectively pushing the layer $m$ to learn the residuum with respect to
the output of layer $n$ rather then the complete golden function.

The rationale behind this change is that by adding layers that perform identity
function we can not worsen (or change at all for that matter) the output of the
network. Therefore, if we take shallow network and increase its complexity by
adding new layers the (training) error should not decrease if the learning to
approximate identity function is easily achieved. However, as the training
error increases in traditional design it can be argued that to learn identity
function is quite hard using traditional gradient descent algorithm. By adding
the residual connection the problem of learning to approximate identity function
became just a task of driving the weights of the connection to zero, which
should be easier that to learn identity function directly. As already mentioned
this rationale was successfully empirically tested on several datasets.

The original authors further improved this design in \cite{resnetimp} by
reorganizing the layers in the network. They removed the activation function
from the residual connection (or, more precisely, they moved it to ``skipped''
part of the \gls{nn}). This way thay simplified the shortcut even more. They
further support this change by thorough empirical testing.
\todo[inline]{obrazok}

\subsection{MobileNets}

\label{ssec:mobilenet}

Other notable architectures are MobileNets by \cite{mobilenets}. In these
architectures authors successfully created models with a relatively low latency while
maintaining high level of accuracy. The motivation behind such architecture is
to provide suitable models for usage in mobile vision application. They
achieved the lowered latency of inference by simplifying some operation in the \gls{nn}.

Most notable change is the replacement of the standard convolution layers as
described in \autoref{sec:conv} by a pair of layers -- depthwise convolutional
filter and pointwise convolution filter.

In the first of these layers -- depthwise convolutional filter -- the
convolution is applied on each input channel separately:
$$y_{i, j, r} = \sum_{k, l} x_{i+k, j+l, r} c_{i, j, r}$$
The latter of these two layers -- pointwise convolution filter -- is just simple
convolution with kernel of size $1 \times 1$.

This replacement of the convolutional layer by two simpler layers significantly
decreases the computation cost of inference, as well
as the time required for training such network. Furthermore the accuracy of
resulting \glspl{nn} was verified on the ImageNet task.

This architecture was then further improved by \cite{mobilenetv2} by
introduction of inverted residuals and linear bottlenecks. As these concepts
are quite complex we refer the reader to the original article for more
information.

\todo[inline]{niekde by som asi este spomenula imagenet v tejto kategorii, lebo pracujes so sietami trenovanymi na nom}
\chapter{Formulation and Workflow}

\label{ch:workflow}

%123456789 123456789 123456789 123456789 123456789 123456789 123456789 123456789
Here we present a formal description of our task and the basic overview of the proposed framework. In this context, we also briefly review the \gls{VL} system as our work is intended to work as its module.

\section{Videolytics}

\label{sec:vl}

Our work is designed to be a module of the analytical system for video processing and advanced analytics by \cite{videolytics}, called \gls{VL}. The goal of the system is to offer an effective way to produce an answer for various queries. Such requests may range from showing an advanced user interface for a live stream to retrieving statistical data from videos stored for offline processing. In order to compute all needed information, the system utilizes various \gls{ml} models. The system may be used to enhance the basic capabilities of security cameras and provide better data that can be utilized in various areas of city planning.

\Gls{VL} is designed as a modular system. This modularity is achieved by using a database as a core component of the system. Modules use the database as the only means of exchanging information. Such a solution makes it easier to add new modules, as their implementations are independent (aside from the \gls{api} defined by the database tables). An additional advantage of this approach is that it makes the data persistent and, therefore, it is possible to re-evaluate any experiments. For the more thorough description of \gls{VL}'s architecture please refer to \cite{videolyticsarch}. 

\section{Task Formulation}

Our goal is to implement the re-identification module for the \gls{VL}. As described in the Introduction, re-identification is, in essence, matching the same object across different frames (that is across different points in time or recordings from different cameras).

\subsection{Input}

\label{ssec:input}

%123456789 123456789 123456789 123456789 123456789 123456789 123456789 123456789
Firstly, let us describe the input of our module. To focus directly on the re-identification task, we use other \gls{VL}' modules to detect an object within the streams and crop out the relevant part of the frame. The input for our part of the system is then just the cropped images alongside additional metadata. We shall refer to this pre-processed input throughout the thesis as \glspl{det}. Let us formally define this notion of \glspl{det}.

Arguably the most important part of the \gls{det} is the image (crop) of the detected object. As \gls{VL} offers the system for selecting axis-aligned minimal bounding boxes of interesting objects, we focus on rectangular images only. We shall define universe of rectangular images $\mathcal{I}$, using standard RGB representation:

\begin{equation}
\mathcal{I} = \bigcup_{\substack{n \in \mathbb{N} \\ m \in \mathbb{N}}} \{0, 1, \ldots, 2^8-1\}^{3mn}
\label{eq:image}
\end{equation}


This representation describes an image as a matrix of pixels where each pixel is further split into triplets, each element describing one channel (red, green, and blue).

We formalize the notion of the source of the original stream (i.e., camera or the video file) where the detection comes from. For the given task, the only relevant information is the dimension of the stream. Aside from that, we assign each camera a unique identification number (to reflect the fact that we can indeed work with two different cameras of the same dimension). We can then formally define the universe of sources as:

$$\mathcal{C} = \{c_i, c_h, c_w \mid c_i \in \mathbb{N}, c_h \in \mathbb{N}, c_w \in \mathbb{N}\}
 = \mathbb{N}^3$$
 
We shall also refer to sources as cameras interchangeably due to consistency with the other work in the area and the underlying implementation.

%123456789 123456789 123456789 123456789 123456789 123456789 123456789 123456789
The final component of the \gls{det} is the additional metadata. For each detection, we record the relative position $x_0, y_0$, of the top left corner of the detection within the stream. Similarly, we denote $x_1, y_1$ relative position of the bottom right corner. Furthermore, we store a \emph{class} by each detection -- representing the detected object's class. Such class is an element of pre-defined vocabulary $V$. Our vocabulary is defined by pre-processing modules of the \gls{VL} systems. It is a relevant subset of classes used in the COCO dataset by \cite{cocodataset}:

$$V = \{\text{``person'', ``motorcycle'', ``car'', ``truck'', ``train'', ``bicycle'', ``bus''\}}$$


The last component of the metadata is a timestamp $t \in \mathbb{R}^+$. This notation allows us to define the universe of metadata~$\mathcal{M}$ as:

$$\mathcal{M} = \{{(x_0, x_1, y_0, y_1) \in [0,1]^4 \mid x_0 < x_1 \land y_0 < y_1}\} \times V \times \N \times \R$$

The first quadruplet represents the position, then a class mentioned, followed by the camera identifier, and the last number represents the timestamp.

Now, we can define the universe of \glspl{det} $\mathcal{D}$ as combination of images and metadata:

$$\mathcal{D} = \mathcal{I \times M}$$

\subsection{Task}

\label{ssec:task}

As usual in \gls{ml} tasks, we formulate our goal via a pair of input and the desired output (so-called ``golden truth''). Here we present an important distinction of our algorithm to the standard re-identification approach presented in the Introduction and \autoref{sec:person_reid}. The result of standard re-identification is a learned function that predicts if two input \glspl{det} are of the same object or not. On the other hand, our approach takes into consideration all presented \glspl{det} at the same time and partitions them based on the displayed object. It essentially works as a form of clustering.

Usually, the set presented on the input of our algorithm is a set of \glspl{det} extracted from simultaneous recordings from the cameras placed close together (i.e., we can expect objects to move from one camera to another or even to be recorded by multiple cameras at the same time). Working in the context of the whole set of \glspl{det} gives our algorithm an extra piece of information during the inference. For example if two \glspl{det} $d_0, d_1$ are visually different but both are similar to a third \gls{det} $d_2$, we may then conclude that $d_0$ and $d_1$ display the same object (as well as $d_2$) which we may not find out if we did not consider $d_2$.


As mentioned, to actually formalize our goal, we consider ``golden'' partitioning~$A$ of given \glspl{det}~$D$. This partitioning represents true re-identification of input \glspl{det}, i.e. an element~$a \in A$ contains all \glspl{det} that display one object. Our algorithm's goal is then to find this golden partitioning based on the input \glspl{det} $D$. Throughout the thesis we shall refer to any partitioning of set of \glspl{det} (both, the ``golden'' partitioning and the imperfect approximation our algorithm generates) as \glspl{iden}. A visualization of these concepts is available in \autoref{fig:structure_schema}.


\begin{figure}
    \centering
    \def\svgwidth{\textwidth}
    \input{img/structure_schema.pdf_tex}
    \caption{Visualization of different camera streams, detections, corresponding trajectories and identities}
    \label{fig:structure_schema}
\end{figure}



To summarize this concept, we shall refer to a pair of set of related \glspl{det} and the associated golden partitioning $A$ as a \gls{ses}. Formally we can define the universe $\mathcal{S}$ of all \glspl{ses} as:

\begin{equation*}
    \mathcal{S} = \left\{D \subset \mathcal{D},\,A \subset \mathcal{P(D)} \mid A\text{ is paritioning of }D \right\}
\end{equation*}

Our algorithm then works in context of one \gls{ses}~$S = (D, A), S \in \mathcal{S}$. A set of \glspl{det}~$D$ is given to the  algorithm and the goal is to recreate the hidden golden identities~$A$. Our algorithm assigns an arbitrary identification number to each identity, therefore we can formalize the desired output as following clustering function~$r_D^*$:

\begin{equation}
\begin{split}
&r_D^*: D \rightarrow \N \text{ s.t. }\\
&(\forall d \in D) (\forall d' \in D) ((r_D^*(d) = r_D^*(d')) \Leftrightarrow (\exists a \in A) (\{d, d'\} \subseteq a))
\label{eq:reidentification}
\end{split}
\end{equation}
Reconstruction of perfect function $r_D^*$ proves to be a hard problem, 
therefore we focus on construction of some approximation $r_D$.

As we previously noted this definition of re-identification task differs from the standard definition where we are given just two \glspl{det} and should output if given \glspl{det} are of the same object or not. However we can easily obtain the standard formulation once we obtain our output function $r_D$ as follows:

$$
r_D' : D^2 \to \{0, 1\} \text{ s.t. } (\forall d \in D) (\forall d' \in D) ((r_D'(d, d') = 1) \Leftrightarrow (r_D(d) = r_D(d')))
$$

Even though we described our algorithm in context of a single session, our goal is to make an algorithm that works well with any \gls{ses}. In order to provide unbiased estimation of the quality of our approach we provide evaluation that was chosen independently of the dataset we used for training \glspl{nn}.

% So far, we have presented our task's definition in the context of already given \gls{ses}. This definition as-is is enough for the work presented in this thesis as we work only with a few selected \glspl{ses}. Nevertheless, we proceed to shortly define the general case in this setting to provide building blocks for future work and motivation for some of our decisions. In a general case, we should not focus on one specific \gls{ses}, but we aim to design an algorithm that works across \glspl{ses}.


% However, there is no algorithm that would output suitable $r_D^*$ as in \autoref{eq:reidentification} for all the \glspl{ses}. To see that we just take any distinct pair of \glspl{det}~$D^\bot = \{d_1, d_2\}$. There exists two distinct partitioning and corresponding \glspl{ses}: $S_1 = (D^\bot, \{\{d_1, d_2\}\})$ and $S_2 = (D^\bot, \{\{d_1\}, \{d_2\}\})$. The output function $r_D$ of any algorithm can satisfy \autoref{eq:reidentification} either for $S_1$ or $S_2$ not for both.


% Therefore, we shall consider only ``possible'' \glspl{ses}. To properly define such \glspl{ses}, we shall assume some distribution $p_\mathcal{S}$ of \glspl{ses} (this distribution should reflect the distribution we aim the algorithm to work on, i.e. distribution of the \glspl{ses} in the real world). We can define possible \glspl{ses}~$\mathcal{S'}$ as:

% $$\mathcal{S}' = \{S \in \mathcal{S} \mid p_\mathcal{S}(S) > 0\}$$
% If every set of \glspl{det} is in at most one \glspl{ses}, i.e.:

% \begin{equation}
%     (\forall (D, A) \in \mathcal{S}')(\forall (D', A') \in \mathcal{S}') (D = D' \Rightarrow A = A'),
%     \label{eq:uniq_session}
% \end{equation}
% Then we can formulate our task by requiring the output function to hold
% condition in \autoref{eq:reidentification} for all
% \glspl{ses} from $\mathcal{S}'$.

% For an (arguably odd) case that \autoref{eq:uniq_session} does not hold, we
% define most likely partitioning for any set of \glspl{det}, given the distribution
% of \glspl{ses}~$p_\mathcal{S}$: 

% \begin{equation}
% A^*_D = \text{argmax}_A \pr_{(D', A')\sim p_\mathcal{S}} \left[A' = A \mid D' = D\right]
% \label{eq:mla}
% \end{equation}

% We can then obtain the valid task definition. We just replace the
% \glspl{iden}~$A$ in \autoref{eq:reidentification} by the most likely
% \glspl{iden}~$A_D^*$ and require the condition to hold over all sets of
% \glspl{det} from possible \glspl{ses}:

% \begin{equation}
% (\forall d \in D) (\forall d' \in D) ((r_D^*(d) = r_D^*(d')) \Leftrightarrow (\exists a \in A_D^*) (\{d, d'\} \subseteq a))
% \label{eq:formulation-final}
% \end{equation}


% Such definition is in agreement with the definition presented for the case where condition in \autoref{eq:uniq_session} holds.


% Finally, let us note that as we picked recordings (and corresponding \gls{ses}) for evaluation of our algorithm randomly, the results of such evaluation reflect the quality of the algorithm with respect to the task as defined by \autoref{eq:formulation-final}.


\section{Workflow}

\label{sec:workflow}

%123456789 123456789 123456789 123456789 123456789 123456789 123456789 123456789

In this section, we describe the general workflow of our re-identification algorithm. Note that, in this thesis, we present several different approaches for the task at hand. This section serves as an overview of the common workflow. Each step is then discussed more in-depth further in the thesis. For the diagram of the workflow, refer to \autoref{fig:workflow}.

\begin{figure}
    \centering
    \def\svgwidth{\textwidth}
    \input{img/workflow.pdf_tex}
    \caption[General workflow of our re-identification algorithm]{General workflow of our re-identification algorithm. Solid lines represent primary, intended dependencies.
    Dashed lines mark secondary, optional dependencies, that some algorithms may use.}
    \label{fig:workflow}
\end{figure}

Firstly, we shall extract the feature vectors from the images of the \glspl{det}. Such a feature vector should sufficiently describe the image as we shall not use the images directly further in the pipeline. There are several advantages of such representation. They primarily stem from the fact that once the feature vectors are extracted, the original images can be discarded. For one thing, we can lower the memory consumption for storing the data if the extracted feature vectors are smaller than the original cropped images. Another advantage is -- assuming that the feature extraction is an irreversible operation -- that it produces a layer of anonymization, which may be crucial for our task as we work with potentially sensitive data. We describe this step in \autoref{ch:features}.

After this feature vector extraction, we proceed to actual re-identification, which we discuss in \autoref{ch:iden_construction}. We usually split this process into two parts.

The first part is to group \glspl{det} that are close together (i.e., they originate from the same camera, have a very similar position within the frame, and have similar timestamp). We refer to this preliminary grouping as \emph{trajectories}. Although some very advanced strategies can be used even for this stage of the grouping, we shall only work with straight-forward approaches and focus more on the second step. We leave the experimenting with more thought-through approaches as feature work.

The second part is then to group generated \emph{trajectories} into \emph{identities}. In this step we need to merge trajectories of the same object across multiple cameras. That makes the spatial metadata less useful. Therefore, we focus more on the visual information in this part.

While this split is a bit arbitrary (and we indeed explore procedures that construct \glspl{iden} directly from \glspl{det} and feature vectors), it arises from practical use. We can use a straight-forward yet precise algorithm while we do not rely on the visual information for the first step. In the second step, we can leverage the fact that we have significantly fewer (by a few magnitudes) objects to merge.
\chapter{Feature Extraction}

\section{Histograms}

\section{Neural Networks}
\chapter{Identity and Trajectory Construction}

%123456789 123456789 123456789 123456789 123456789 123456789 123456789 123456789

In this chapter we focus on actual techniques how to group detections into
desired clusters, i.e. creating \glspl{iden} as per \autoref{ssec:task}.
We show how we combine visual information of the image
(\autoref{ch:features}) with metadata to provide \glspl{iden}.

\section{Trajectories}

In this section, we focus on grouping \glspl{det} together by the object they
capture without the visual information, i.e. purely based on the metadata. In
other words we create trajectories as per \autoref{sec:workflow}. By its nature
the \glspl{det} within single trajectory are from consecutive frames and
physically close together.

The reason for this step is to group the \glspl{det} together in cases
where the visual information --- which processing may be complicated --- is not
needed. They then further serve as a ``building blocks'' for subsequent
processes.


% The trajectories serve as a ``building blocks'' for more
% complicated \glspl{iden}. Due to the way they are constructed the trajectories
% are clusters of \glspl{det} that are physically close together. They are usually
% just list of \glspl{det} from the consecutive frames with just a minor changes
% in position of tracked object in-between the frames.

\subsection{Detection Metadata}

Let us briefly recall available metadata for each \gls{det} as introduced in
\autoref{ssec:input}:

\begin{itemize}
    \item $(x_0, x_1, y_0, y_1)$ -- Position of the bounding box of the detection given by $x$-coordinate of the left and the right edge and the $y$-coordinate of top and bottom edge respectively.
    \item $v$ -- Class of the detection, e.g. person, car or truck
    \item $c$ -- Identificator of a the camera
    \item $t$ -- Timestamp
\end{itemize}


Let us start with a simplification. We consider metadata of just two
\glspl{det} at hand, and we explore if we can find out whether the \glspl{det}
display the same object. We discuss this \gls{iden} correspondence for each part
of the metadata separately.

\subsection{Processing of simple metadata information}

Let us start with the metadata regarding the class of the object. The class
gives us the most straight-forward conclusion.
If the two \glspl{det} at hand are of different classes we know for sure
that they can not be of the same \gls{iden} (i.e. displaying the same
object).\footnote{There is quite
odd case where this seemingly obvious conclusion can be argued against. That
is, for example, when a person enters a car. Based on the actual definition of
the \gls{iden} we may consider it the same displayed object. As these cases
raises multiple questions (e.g. what should happen when multiple people enters the same vehicle), we shall
consider person entering the vehicle and the vehicle itself as different
\glspl{iden} and we leave further research in this topic to
future work.} As a corollary, we therefore assume that the two \glspl{det} are
of the same class while investigating the rest of the metadata (as in the
opposite case we reject that they are of same object).

We can incorporate identificator of the camera just as easily. If the two
\glspl{det} are from different cameras, we do not explore the remaining
metadata, as they do not hold any useful information in such
case.\footnote{There
are approaches that assume that the two given cameras capture largely the same
camera and are calibrated (for example \cite{hu2006principal}). However, we
assume no prior calibration and therefore we leave similar approaches to other
work} For example even if there is one detection in top left corner of one
camera, and in exact same time there is another detection in top left corner
of another camera at the same time, we can not say if the \glspl{det}
display the same person or not. This is
because we do not know if the area within top left corner is the same as the
area in the top left corner of the second camera. In such case we have to use
a priori assumption and assign them to two different
trajectories.\footnote{Unless we have any additional information the a priori
assumption that two detections belong to the same golnden \gls{iden} is more
likely to be true than the opposite if there is no
golden trajectory that would be represented by more than a half of the
\glspl{det} within given \gls{ses}, which seems like a reasonable assumption to
have}

For assignment into the same trajectory we therefore consider only \glspl{det}
from the same camera. Therefore we only elaborate on such cases in the remainder
of this subsection. We regard cases how to discover the same object across
different cameras later in this chapter (\autoref{sec:generating_identities}).

% Nevertheless, it still may be true that the \glspl{det} display the same
% object and we may discover it by processes described later in this chapter.
% However, for the purpose of generating trajectories we process the information
% about camera as described and for the remainder of this section we assume the
% case where the \glspl{det} are from the same camera.

Lastly, us consider additional information that we gain by using the timestamp.
With combination with the spatial data the timestamp can be helpful. However as
we mentioned in the intro of this section
we are now interested in creating short trajectories (i.e. clusters of
\glspl{det} that are physically close together). Therefore
our approach to temporal information is be simple: If the time difference is
bigger than selected threshold (say 1 second, however we experiment with
various settings), we assign the \glspl{det} to different trajectories. In the
opposite case we shall further evaluate secondary conditions on the spatial
information.

\subsection{Conditioning on spatial information}

\label{ssec:spatial_merging}

Let us derive a couple of quantities that may be useful for us when deciding
if we should assign given \glspl{det} to same trajectory. Deciding if we should
assign them to the same trajectory is then just a matter of comparing one or
more of the following quantities with pre-selected threshold. If we find pair
of detection where all the considered quantities are less than corresponding
threshold, we shall merge the trajectories together.

For brevity in the following lines we shall use height ($h = y_1 - y_0$) and
width ($w = x_1 - x_0$) of a detection.

\subsubsection{Displacement}

One of the obvious quantities to consider when comparing two \glspl{det} is
their displacement. We shall compute the displacement with respect to centers
of the \glspl{det}. Formally we can compute a displacement as follows
(superscripts represent which \gls{det} the values are from)

\begin{align*}
    c^{(A)} &= \left(\frac{x_0^{(A)} + x_1^{(A)}}{2}, \frac{y_0^{(A)} + y_1^{(A)}}{2}\right) \\
    c^{(B)} &= \left(\frac{x_0^{(B)} + x_1^{(B)}}{2}, \frac{y_0^{(B)} + y_1^{(B)}}{2}\right) \\
    d &= \delta_{euclid}(c^{(A)}, c^{(B)})
\end{align*}

\subsubsection{Relative Displacement}

However, such simple displacement only express the displacement within the
screen of the camera. If an object close to the camera moves, this movement
will translate to large movement within the screen of the camrea, but if we
perceive the same movement by an object further from the camera, the percieved
movement will be much greater. Luckily, we can estimate the ``closeness'' to the
camera without the prior calibration. Simply by considering the size of the
bounding box of the detection (i.e. bigger the bounding box, more likely it is
that the object is closer to the camera). This works especially if the project
are usually of the same size (i.e. all people). While this estimation is qutie
vague, it has other advantages. For example it also ``expects'' smaller objects
(i.e. kids) to make smaller movements than bigger ones. Nevertheless, this leads
us to the definition of relative displacement where we divide the displacements
by average edge length of the bounding boxes:

$$\frac{d}{\left(h^{(A)} + w^{(A)} + h^{(B)} + w^{(B)}\right) / 4}$$

\subsubsection{Intersection over Union}

The last quantity we shall consider when merging \glspl{det} to trajectories is 
so-called intersection over union. It is simply area covered by both boundign
boxes (i.e. union) divided by area covered by both bounding boxes (i.e.
intersection). For visual interpretation see \autoref{fig:iou}. Formally we can
define the quantity for our rectangular bounding boxes as:

\begin{align*}
    \text{InterWidth} &= \min\left(x_1^{(A)}, x_1^{(B)}\right) - \max\left(x_0^{(A)}, x_0^{(B)}\right) \\
    \text{InterHeight} &= \min\left(y_1^{(A)}, y_1^{(B)}\right) - \max\left(y_0^{(A)}, y_0^{(B)}\right) \\
    \text{Intersection} &= \begin{cases}\text{InterWidth} \cdot \text{InterHeight} & \text{if InterWidth} > 0 \land \text{InterHeight} > 0 \\ 0 & \text{else}\end{cases} \\
    \text{Union} &= w^{(A)} h^{(A)} + w^{(B)} h^{(B)} - \text{Inter} \\
    \text{IoU} &= \frac{\text{Intersection}}{\text{Union}}
\end{align*}

\begin{figure}
    \centering
    \includegraphics[width=6cm]{img/Intersection_over_Union_-_visual_equation.png}
    \caption[Visualization of intersection over union]{Visualization of intersection over union\\Source: Intersection over Union -- visual equation\footnote{\url{https://commons.wikimedia.org/wiki/File:Intersection_over_Union_-_visual_equation.png}} by Adrian Rosebrock, CC BY-SA 4.0\footnote{\url{https://creativecommons.org/licenses/by-sa/4.0}}, via Wikimedia Commons}
    \label{fig:iou}
\end{figure}

\subsection{Trajectory generation}

\label{ssec:trajectory_generation}

In this section we show how we generate local trajectories based on
the information overviewed in previous section. Comparing every \gls{det} with
every other \gls{det} takes quadratic time with respect to number of detection
and for hundredths of thousands of \glspl{det} is unfeasible to compute on
standard hardware. Therefore, we need to lower the number of comparison.

We leverage the way we incorporated timestamp into trajectory generation
we described in previous subsection. By considering only
\glspl{det} under given threshold allows us to compare only \glspl{det} by using
sliding window. We shall process the \glspl{det} in order of their timestamp.
When we process a new \gls{det} we add it to the sliding window. At the same
time we remove all the \glspl{det} from the window that are older than the
timestamp of the new detection minus the threshold. Then we compare the new
\gls{det} with all the \glspl{det} within the window. This dramatically decrease
the number of comparison compared to approach where we compare all the
\glspl{det} with every other \gls{det}.

However, this approach does not mean that the length of the trajectories is
bounded by the given time threshold. We use ``transitive'' information about
the trajectory. Namely, we know it a \gls{det} (A) belongs to the same
trajectory as \gls{det} (B) and \gls{det} (B) belongs to same trajectory as 
\gls{det} (C), we can conclude that the \gls{det} (A) and \gls{det} (C) is of
same trajectory. We shall use standard union-find (originally described by
\cite{galler1964improved}) architecture which allows us to track which \glspl{det} belongs to which trajectory and merge them in almost
constant\footnote{The actual complexity of the operation is inverse
Ackermann function of the size of the structure per query (amortized) as were
proved by \cite{tarjan1984worst}. For the details of the structure please refer
to the attached code or the original research.} time per request.
Furthermore, such approach allows us
to set the threshold to stricter values and still capture via ``transitive''
property the full length of the trajectory.

\section{Identities}

\label{sec:generating_identities}

So far we have looked into how to construct trajectories.
There are just a few steps to merge these trajectories together to fulfill
our original goal. That is to obtain clusters of detection where each cluster
contains all the detection display the same object -- i.e. what we define
as \gls{iden}.
\todo{Preformulovat}

\subsection{Trajectory merging}

We merge the trajectories to bigger \glspl{iden} by incorporating the visual
information. The rationale is that we used all metadata which allowed us to
cluster \glspl{det} that were physicially close together to trajectories.
Now we make use of the visual information we have in form of the feature vectors
(generated by approaches described in \ref{ch:features}) to connect \glspl{det}
that may be physically distant or recorded on different cameras to obtain
\glspl{iden}.

The key information in this step is the distance between the feature
vectors of \glspl{det}. We shall explore multiple distance functions
(Euclidean, Manhattan, Cosine -- for definitions please refer to
\autoref{sec:distances}) and their effect on the quality of resulting
\glspl{iden}.

The straight-forward approach is to use the same algorithm as when we
generated the trajectories, except that use the feature vectors
rather than metadata. In this approach we decide if we marge two trajectories
together by comparing feature vectors of their \glspl{det} and their distance
with respect to selected distance function. If for any pair of \glspl{det} 
(one \gls{det} from each trajctory) the computed distance is lower then selected
threshold, we merge the trajectories together. This requires to compute distance
between each \gls{det} from the first trajectory with each \gls{det} of the
second trajectory. Hence this again leads to quadratic complexity.
Therefore, we present a way to decrease this number significantly.

\subsubsection{Representant Selection}

We present a relatively simple approach to decrease number of comparison needed.
The idea behind the approach is to select from each trajectory a set of
representants. We want to select a  suitable set of representsnts that will
preserve all information needed for creating appropriate \glspl{iden}.

To pick the representant effectively we make a simple observation. Let there be
two different \glspl{det} within a trajectory, that are fairly similar. Then
whenever we want to decide if we want to add another \gls{det} into a trajectory
we need to compare the \gls{det} to just one of the original \gls{det}. The
point is that while the measured distance may be grater than when compared
to the other of two \glspl{det}, due to the triangular inequality, the distance
will be greater at most by the distance between two original \glspl{det}.
As noted this observation depends on triangle inequality and thus holds only
for metric functions.
For systematic way of how to select representnts using this observation
see \autoref{alg:representants}.

We can apply this observation in a context of more than just three \glspl{det}.
Let us assume that we  start with trajectories that are ``clean'' (i.e.
trajectory contain \glspl{det} of just a single golden \gls{iden}). Further
assume that the golden \glspl{iden} are ``well-separated'' using the selected
metric function in a following sense: The minimal distance between two
\glspl{det} from different golden \glspl{iden} is at least $\Delta_{mrg}$
and for every pair of \glspl{det} from any \gls{iden} there exists a sequence
of \glspl{det} such that every \gls{det} is from the same \gls{iden} and
and distance between two consecutive \glspl{det} is at most
$\Delta_{mrg} - 2\Delta_{rep}$. Then the produced \glspl{iden} produced by
selection representants by \autoref{alg:representants} and then merging
trajectories that have \glspl{det} closer than $\Delta_{mrg}$
are always exactly golden \glspl{iden}.


% The observation that will help us select proper representative is that if
% there are two \glspl{det} within a trajectory, that are similar (in terms of
% distance between their feature vectors), than whatever third \gls{det} is
% similar to the first \gls{det} is similar also to second \gls{det} (at least
% as long as we use metric and not generic distance as we show in a while).
% Therefore if there is a pair of similar \glspl{det} in a trajectory, we select
% at most one as a representant and discard the other. For systematic way of
% selecting representant leveraging this property see \autoref{alg:representants}.


\begin{algorithm}
 \SetKwInOut{Input}{input}
 \SetKwInOut{Output}{output}

 \Input{Trajectory $T$ (as a list of feature vectors of detections), Threshold $t$}
 \Output{List of feature vectors of representants}
 
 \BlankLine
 $R \leftarrow$ \emph{empty list}\;
 \For(\tcp*[h]{outer loop}){$D_{new} \in T$}{
  \For{$D_{rep} \in R$}{
   \If{$\delta(D_{new}, D_{rep}) \leq t$}{
    skip on next iteration of outer loop\;
   }
  }
  append $D_{new}$ to $R$\;
 }
 \Return $R$
 \caption{Selection of representants of a trajectory}
 \label{alg:representants}
\end{algorithm}

% Our intuitive observation, that we need to preserve only one of the two similar
% \glspl{det} in a trajectory can be formalized as:

This claim can be formalized as follows:

\begin{figure}
    \centering
    \def\svgwidth{\columnwidth}
    \input{img/claim.pdf_tex}
    \caption[Visualization of \autoref{clm:claim}]{Visualization of  \autoref{clm:claim}. The figure shows two \glspl{iden}, first of them dividen into three trajectories, latter into two. White dots represents \glspl{det}, black dots represent \glspl{det} which were chosen as representants. Solid lines (shown only is the first trajectory) shows shortest distance to a representant, such distances are are most $\Delta_{rep}$. Dashed lines shows distances between representants, such distances are greater than $\Delta_{rep}$. Dotted lines shows some distances between \glspl{det} of different trajectories of the same \gls{det}. The claim considers situations where all such distances are at most $\Delta_{mrg} - 2\Delta_{rep}$. If such distance between two representants are at most $\Delta_{mrg}$, the corresponding trajectories will be merged together (i.e. they are in $\circ$ relation. The grey lines are distances between \glspl{det} of different \glspl{iden}, the claim considers cases where such distances are greater than $\Delta_{mrg}$}
    \label{fig:claim}
\end{figure}

\begin{claim}
\label{clm:claim}
Let:

\setlength{\itemsep}{0pt}
\setlength{\parskip}{0pt}

\begin{itemize}
    \item $D$ be a set of \glspl{det}
    \item $A$ a partitioning of $D$ (this represents golden \glspl{iden})
    \item $T$ partitioning of $D$ such that $(\forall t \in T) (\exists a \in A) (t \subseteq a)$ (this represents input ``clean'' trajectories)
    \item $\Delta_{rep}, \Delta_{mrg} \in \R^+$ such that $\Delta_{rep} < \Delta_{mrg}$
    \item $\delta$ be an metric function over $D$
    \item $r$ be a function $r: \Pt{D} \goto \Pt{D}$ such that $(\forall c \subseteq D) (\forall b \in c) (\exists b' \in r(c)) (\delta(b, b') \leq x_{rep})))$ (this represents selection of representants obtained for example via \autoref{alg:representants})
    \item $\circ$ be a relation over $T$ such that $t\circ t' \Leftrightarrow (\exists d \in r(t)) (\exists d' \in r(t')) (\delta(d, d') \leq x_{mrg})$ (two trajectories are in relation $\circ$ if we can merge them together)
    \item $\bullet$ be a transitive closure of $\circ$ (i.e. $t \bullet t' \Leftrightarrow (\exists n \in \N) (\exists (t_1, \ldots, t_n) \in T^n) (t = t_1 \land t' = t_n \land (\forall i) (t_i\circ t_{i+1}))$ (this relation is equivalence that represents final \glspl{iden} -- two trajectories are in relation $\bullet$ if and only if they were merged together)
\end{itemize}

If the following holds (i.e. the golden \glspl{iden} are ``well-separated''):
\begin{itemize}
    \item $(\forall a \in A)(\forall d, d' \in a) (\exists n \in \N) (\exists (d_1, \ldots, d_n) \in D^n) (d = d_1 \land d' = d_n \land (\forall i) (\delta(d_i, d_{i+1}) \leq x_{mrg} - 2x_{rep}))$
    \item $(\forall a, a' \in A) (\forall d \in a) (\forall d' \in a') (a \neq a' \Rightarrow \delta(d, d') > x_{mrg})$
\end{itemize}

Then: $t\bullet t' \Leftrightarrow (\exists a \in A) (t \cup t' \subseteq a)$ (i.e. produced \glspl{iden} exactly match golden \glspl{iden})

% Let $D$ be a set of \glspl{det}, $A$ its partitioning, $\delta$ a metric over
% $D$, $T$ another partitioning of $D$ such that $(\forall t \in T) (\exists a \in A) (t \subseteq a)$ and thresholds $x_{rep}, x_{merge} \in \R^+$. Denote $r: \Pt{D} \goto \Pt{D}$ generator representants as per \autoref{alg:representants} with threshold $x_{rep}$. Now consider $R$ to be a relation over $T$ such that $tRt' \Leftrightarrow (\exists d \in r(t)) (\exists d' \in r(t')) (\delta(d, d') \leq x_{merge})$ and let $\widehat{R}$ be its transitive closure.

% If $\min_{a, a' \in A : a \neq a'} \min_{d \in a, d' \in a'} \delta(d, d') > x_{merge}$ and
% $(\forall a \in A)(\forall d, d' \in a)(\exists (d = d_1, d_2, \ldots, d_n = d') \subseteq a)(\forall i)(\delta(d_i, d_{i+1}) \leq x_{merge} - 2x_{rep})$ 
% then $t\widehat{R}t' \Leftrightarrow (\exists a \in A)(t \cup t' \subseteq a)$.

\end{claim}

% This claim basically tells us that if we start with the trajectories that
% are ``clean'' (i.e. each trajectory is a subset of some golden \gls{iden}) and
% the \glspl{iden} are well-separated (i.e. distance between golden \glspl{iden}
% are at least $x_{merge}$ and for any pair of the \glspl{det} within a \gls{iden}
% there is a sequence of \glspl{det} such thateach subseqent pair of \glspl{det} within the sequence is distant at most $(x_{merge} - 2x_{rep})$ and the ends
% of the sequence are the original two \glspl{det}, then the produced clustering
% of the \glspl{det} corresponds to the desired golden \glspl{iden}.

\begin{myproof}
First, let us prove that $t \bullet t' \Rightarrow (\exists a \in A) (t \cup t' \subseteq a)$. We shall prove this by contradiction, i.e. for now we assume
that $(\exists a \in A) (t \in a)$ and $(\exists a' \in A) (t' \in a')$ and $a \neq a'$. From property of $\bullet$ we know there is a sequence of $t_1, t_2, \ldots t_n$ (for some $n$), such that $(\forall i) (t_i \circ t_{i+1})$. Let
us take minimal $i$ such that $t_i$ and $t_{i+1}$ is not subset of the same
$a \in A$ (such $i$ exists as $t$ and $t'$ does belong to different $a$s).
For such $i$ (since $t_i \circ t_{i+1}$) there is $d \in t_i$ and
$d' \in t_{i+1}$ such that $\delta(d, d') \leq x_{mrg}$. However since $d \in a$
and $d' \in a'$ (such that $a \neq a'$) this is in contradiction with second
of the original assumptions.

Now let us prove that $(\exists a \in A) (t \cup t' \subseteq a) \Rightarrow t \bullet t'$. Let us choose $d \in t$ and $d' \in t'$ arbitrarily. By the
assumptions for elements of $a$ there is sequence $(d_0, \ldots, d_n) \in A^n$
such that $d = d_0 \land d' = d_n \land (\forall i) (\delta(d_i, d_{i+1}) \leq x_{mrg} - 2x_{rep}$. Now let us denote for each $i$ $t_i \in T$ such that
$d_i \in t_i$. Finally, let us denote $r_i = \mathrm{argmin}_{d \in r(t_i)} \delta(d, d_i)$. If $t_i = t_{i+1}$ then $t_i \circ t_{i+1}$ by definition.
So let us consider cases where $t_i \neq t_{i+1}$. In such case we can bound
the distance between $r_i$ and $r_{i+1}$ due to the triangle inequality:
$\delta(r_i, r_{i+1}) \leq \delta(r_i, d_i) + \delta(d_i, d_{i+1}) + \delta(d_{i+1}, r_{i+1}) \leq x_{rep} + (x_{mrg} - 2x_{rep}) + x_{rep} = x_{mrg}$. And as $r_i \in r(t_i)$ we can conclude that $(\forall i) (t_i \circ t_{i+1})$. Therefore $t \bullet t'$.\end{myproof}


\begin{cor}
If the distance between two closest \glspl{det} from different golden
\glspl{iden} is $\Delta_{between}$ and the distance between two most distance detection
of the same golden \gls{iden} is $\Delta_{within} < \Delta_{between}$, then for any set of ``clean''
trajectories (i.e. if each trajectory contains \glspl{det} only from single
golden \gls{iden}) the \autoref{alg:representants} for representant selection
with threshold $(\Delta_{between} - \Delta_{within} - \varepsilon) / 2$ (for arbitrarily small
$\varepsilon \in \R^+$) followed by merging based on representants with
threshold $\Delta_{between} - \varepsilon$ produces the golden \glspl{iden}.
\end{cor}

Let us note that the original assumption is fulfilled if the Triplet Loss
over the examinated dataset is zero (and in some cases even if it is positive).
This is usually too strong assumption to make, however it still gives us some
theoretical justification for our approach.

In case that $\delta$ is not metric function but just a generic distance
function we have no such guarantees. However we still evaluate this approach
even for such distance functions.

\subsection{Direct identity generation}

So far we have presented two-step approach. In the first step we generate
trajectories and during the second step we merge these trajectories into
final \glspl{iden}.

However, we also experiment with the approach where we
generate the identities directly. This approach would work similar as to
original trajectory detection where we have a sliding window of \glspl{det}
and we merge two sets of \glspl{det} together if we find two similar
\glspl{det}. This approach has significant drawback of not being able to
merge two sets of \glspl{det} together if there is too big ``time gap'' between
two consecutive \glspl{det}. Nevertheless, we believe it is worth to experiment
with this direct approach.

However, we still want to be able to at least assign two \glspl{det} to the
same \gls{iden} if they both fit into the sliding window, even though they are
from different cameras. Therefore, our approach will be as follows: If we are
considering two \glspl{det} for merging, we evaluate the spatial information
exactly as in \autoref{ssec:spatial_merging}. If all the conditions are met,
then we merge corresponding set of \glspl{det}. If one of the threshold is
exceeded then we evaluate the visual information. If the corresponding feature
vectors are closer then pre-selected threshold then we do the merge even though
the information from metadata is not conclusive.

\section{Recapitulation}

In this chapter we elaborated on two ways of generating \glspl{iden} that we
experiment with (trajectory generation with subsequent merging and direct
approach). Both are dependent on which feature vectors will be used (which we
described in \autoref{ch:features}). Furthermore, each approach is dependent
on selection of various thresholds, we experiment with various setting of those
thresholds. The list of quantities with threshold is:

\begin{itemize}
    \item Size of the sliding window
    \item Displacement
    \item Relative displacement
    \item Intersection over Union
    \item Visual distance
\end{itemize}
\chapter{Evaluation}

%123456789 123456789 123456789 123456789 123456789 123456789 123456789 123456789

\label{ch:evaluation}

In this chapter we aim to empirically evaluate the approaches we have described
in previous chapters of this work.

\section{Datasets}

\label{sec:datasets}

In this thesis we work with two distinct \glspl{ses}.

The first \gls{ses} comes from processing recording from single camera which
capture a busy square. The original recording is about 5 minutes long. The total
number of \glspl{det} extracted from the recording is 181,792. We manually sorted
these \glspl{det} into 274 different \glspl{iden}. The main usage of this
\gls{det} is to be a training dataset for training neural network based
approaches to extract feature vectors from the images of \glspl{det}. The
manual annotation serves as a golden truth for this training. A frame from
this \gls{ses} and extracted \glspl{det} can be seen in
\autoref{fig:single_session}.

\begin{figure}
    \centering
    \includegraphics[width=0.48\textwidth]{img/frame_single_session_smaller.png}
    \includegraphics[width=0.48\textwidth]{img/frame_single_session_det_smaller.png}
    \caption{Frame from first session and extracted detections}
    \label{fig:single_session}
\end{figure}

We use the second \gls{ses} for actual evaluation. This second session is
constructed from the recording of two simultaneously recording camera. The
recording is over two minutes long. The view of the cameras partially overlap.
In this session total of 147,681 \glspl{det} were extracted. We sorted these
\glspl{det} into 52 unique \glspl{iden}. These \glspl{iden} were used as a
ground truth for actual evaluation of our approach. Examples of frames from
this \gls{ses} is in \autoref{fig:double_session}.

\begin{figure}
    \centering
    \includegraphics[width=0.48\textwidth]{img/frame_double_session_1_smaller.png}
    \includegraphics[width=0.48\textwidth]{img/frame_double_session_2_smaller.png}
    \caption{Frames from second session}
    \label{fig:double_session}
\end{figure}

The recordings for those two \glspl{ses} are entirely disconnected -- they
capture entirely different place and were recorded at different time. That gives
us unbiased estimation of the quality of our algorithm in terms of
\autoref{eq:mla}. It would be better to evaluate our algorithm to multiple
\glspl{ses}, however due to technical difficulty of creating new \gls{ses}
(especially in terms of annotating the data), we leave more thorough evaluation
to feature work.

\section{Feature vectors}

As the first step of actual evaluation of our approaches we evaluate the
quality of feature vectors as described in \autoref{ch:features}. We shall
evaluate this part purely based on how well the resulting feature vectors
separate the golden \glspl{iden} from each other. We recognize there is some
information lost, as we skip the final step in this evaluation -- that is to
construct actual \glspl{iden} and compare them with the golden \glspl{iden}.
However, this preliminary evaluation is computationally less expensive and
therefore we may experiment with broader spectrum of parameters. We shall
select couple of best performing setting and verify its usefulness in later
stages.

\subsection{Measures of Quality}

This preliminary testing offer simpler statistics for evaluation. In this scenario we need to just look into distances between \glspl{det} of the same golden \gls{iden} and the distances between \glspl{det} of distinct golden \glspl{iden}. We can further associate each approach with a threshold. The ideal scenario would be that the maximal distance between \glspl{det} from the same \gls{iden} would be at most this threshold and all the distances between \glspl{det} from distinct \glspl{iden} would be greater than the threshold. However, in most cases we do not have such ideal classifier. Thus we often need to deal with misclassified samples.

Overall we can assign every pair of detection in on of four standard categories. First two are \glspl{tp} and \glspl{tn} they represent correct classification -- pair of \glspl{det} from the same \gls{iden} with distance between them below selected threshold and pair of \glspl{det} from distinct \glspl{iden} with distance over the threshold respectively. The other categories are \glspl{fp} -- pairs that are of different \glspl{iden} but with distance below the threshold -- and \glspl{fn}, that are pairs of \glspl{det} from the same \gls{iden} but with the distance over the threshold.

This notation allows us to introduce two standard derived quantities --- \gls{tpr} and \gls{fpr}:

\begin{align*}
    \mathrm{TPR} &= \frac{\mathrm{TP}}{\mathrm{TP} + \mathrm{FN}} \\
    \mathrm{FPR} &= \frac{\mathrm{FP}}{\mathrm{FP} + \mathrm{TN}}
\end{align*}

Notice that when we increase the threshold, increase the number \glspl{tp} and \glspl{fp} and decrease number of \glspl{tn} and \glspl{fn}. In terms of rates, that means that both \gls{tpr} and \gls{fpr} increases. This allows us to draw a plot of how \gls{tpr} depends on \gls{fpr}. That is a quite common way to elaborate on the quality of \gls{ml} algorithm. The resulting plot is often called \gls{roc} curve.

Ideal classifiers would have a \gls{roc} curve going through a point
(0, 1). Such classifier with corresponding threshold would correctly
classify every input, positive or negative. We usually do not
have access to ideal classifiers. In such case we are, broadly speaking,
want to come up with classifier that have \gls{roc} curve as close to
point (0,1) as possible.

However, \gls{roc} curves gives us more in-depth view into the classifier.
The curve shows which trade-offs between \glspl{fp} and \glspl{fn} (or in terms
of the axis \gls{fpr} and \gls{tpr}) are possible for given classifiers.

While the \gls{roc} curve offers quite useful evaluation of the
selected approach, it is often useful to express the quality of the
approach as a single number. Some information will be lost this way,
on the other hand it offers very straight-forward way to compare different
classifiers. For this evaluation we leverage also leverage the \gls{roc} curve.
In particular we compute the area under the curve. Generally, the greater
the area the better classifier we have.

Finally, let us note that we want to get feature vectors that help us to find the same object in physically distant \glspl{det}. Therefore, we are not really interested in the distance between \glspl{det} of the same \gls{iden} from subsequent frames (as they would be matched by metadata anyways without the visual information). Therefore, for this preliminary evaluation (and drawn \gls{roc} curves) out of all ``positive'' pairs (i.e. pairs of \glspl{det} from the same \gls{iden}) only those which are either from different cameras or are at least 2 seconds apart.

\subsection{Evaluation of Color Histograms}

Firstly, we aim to evaluate various approaches for feature vector generation
based on the color histograms. To recapitulate there are several basic choices
for histogram generation we need to evaluate (please refer to
\autoref{sec:histograms} for detailed explanation):

\begin{itemize}
    \item Selection of color model
    \item Type of background filtering
    \item Number of bins of a histogram
    \item Choice of distance function
\end{itemize}

\subsubsection{Evaluation of Background Filtering}

As there are too many options per category to evaluate all combinations we
firstly evaluate which background filtering mechanism works best for us. As we
aim to decide what section of the crop is important for the re-identification
the selection of background filtering should be relatively independent of
the choice of color model and number of bins. For the experiments in this
subsection we selected a hue and saturation with histogram with 8 bins in
each component (i.e. 64 bins total).

Let us recall that we reviewed several approaches to background filtering.
Each approach allows for fine-tuning via various parameters:

\begin{itemize}
    \item No background filtering -- no additional parameters
    \item Filtering using cropping -- total of three parameters: percentage of the image cropped from left and right, percentage of the image cropped from the bottom, and percentage of the image cropped from the top
    \item Weighting by Gaussian -- total of two parameters: center of the Gaussian along $y$-axis and the scale of the Gaussian
\end{itemize}

For each of these approaches we use grid search to find optimal values of
these parameters. The main value we use for comparison is the size of the
area under the \gls{roc} curve. However, we still explore some of the
\gls{roc} curves directly.

\begin{figure}
    \centering
    \def\svgwidth{\columnwidth}
    \Large
    \scalebox{0.6}{\input{img/aoc_crop_30.pdf_tex}}
    \scalebox{0.6}{\input{img/aoc_crop_20.pdf_tex}}
    \scalebox{0.6}{\input{img/aoc_crop_40.pdf_tex}}
    \caption{Area under the ROC curve for various copping setting}
    \label{fig:aoc_crop}
\end{figure}

As we can see from the measurements in \autoref{fig:aoc_crop} any method of background filtering significantly improved the quality of the feature vectors. We note that we achieved the best performance with cropping when we 
cropped 30\% image from the left and right side, 20\% from the top and 30\%
from the bottom (although variant with 20\% achieved almost as good results). An example of such cropping can be seen in
\autoref{fig:best_cropping}.

\begin{figure}
    \centering
    \includegraphics{img/0.png} \includegraphics{img/1.png} \\
    \includegraphics{img/0_crop.png} \includegraphics{img/1_crop.png}
    \caption{Example of most effective cropping}
    \label{fig:best_cropping}
\end{figure}

\begin{figure}
    \centering
    \def\svgwidth{\columnwidth}
    \Large
    \scalebox{0.6}{\input{img/aoc_gauss.pdf_tex}}
    \caption{Area under the ROC curve for various setting of Gaussian weighting}
    \label{fig:aoc_gauss}
\end{figure}

In terms of weighting with Gaussian, we have achieved the best results
by offsetting the Gaussian slightly above the center of the image, to 40\%
of the height of the image to be precise. The best setting for the scale
of the Gaussian seems to be 0.2 or 0.1 of the dimensions of the image. For detailed results
of the grid search see \autoref{fig:aoc_gauss}.

As we can see in \autoref{fig:roc_background} all the selected approaches
with background filtering gives almost identical results.

\begin{figure}[tb]
    \centering
    \def\svgwidth{\columnwidth}
    \input{img/background_roc.pdf_tex}
    \caption{ROC curve of various type of background filtering}
    \label{fig:roc_background}
\end{figure}

\subsubsection{Choice of distance function}

Another subject of our experiments is the choice of distance function. In
\autoref{ssec:used_distances} we introduced three distance functions --
Euclidean, Manhattan and cosine. As we can see in \autoref{fig:roc_distances}
the choice of distance function have significant effect on the results.
The worst distance function seems to be Euclidean distance. The best one,
especially while preserving lower \gls{fpr} seem to be Manhattan distance.
After all, Manhattan distance has quite straigh-forward explanation in context
of histograms -- it is simple the amount each bin needs to decrease or increase
in order to achieve the second histogram.

\begin{figure}
    \centering
    \def\svgwidth{\columnwidth}
    \input{img/roc_distances.pdf_tex}
    \caption{ROC curve of various distance functions}
    \label{fig:roc_distances}
\end{figure}

\subsubsection{Selection of Color Model and Number of Bins}

The last parameter we need to appropriately set is the actual color model
and number of bins per histogram. We explore several color models:

\begin{itemize}
    \item RGB model
    \item HSV model (we select hue component for one histogram and both hue and saturation component for another one)
    \item YUV model (we select UV components)
\end{itemize}

\begin{figure}
    \centering
    \def\svgwidth{\columnwidth}
    \Large
    \scalebox{0.7}{\input{img/model_bins.pdf_tex}}
    \caption[Area under the ROC curve for various color models and number of bins]{Area under the \gls{roc} curve for various color models and number of bins. Number of bins are displayer per channel. Total number of bins for HS and UV is number of bins per channel squared and for RGB it is number of bins per channel cubed. For some combinations were not explored as a total number of bins for these combination requires too much memory stored per each feature vector.}
    \label{fig:model_bins}
\end{figure}

For each model we explore various numbers of bins. The sizes of areas under the corresponding \gls{roc} curves is displayed in \autoref{fig:model_bins}. Contrary to our original
assumption the results gives usage of the raw RGB channels. The best variation in particular was with 4 bins per channel (i.e. 64 bins total).

This can be explained by various phenomena For one the varying lightning
conditions are not as common in our dataset as we expected. The other aspect
is that as we notice large quantity of the images are of people with black or
dark clothing. The downside of using hue as a component in histograms is that
hue can change very significantly case of black and white colors even in small
changes in actual color. See \autoref{fig:bad_hue} for the visualization.

\begin{figure}
    \centering
    \includegraphics[width=3cm]{img/bad_hue_orig.png}
    \includegraphics[width=3cm]{img/bad_hue_hue.png}
    \caption[Hue extraction from an image]{Hue extraction from an image. The image on the left is original. The right image was obtained by preserving hue of each pixel but setting saturation and value to the same high level. As we can see, the mono-colored black coat has various hue levels and on the other hand the black coat and the white background is represented by similar hue levels even tough both are of entirely different colors.}
    \label{fig:bad_hue}
\end{figure}

We aim to support this reasoning by considering pixels with low ($< 0.2$)
value as black and non-black pixels with low saturation ($< 0.2$) as white and
assign them to separate bins. As we can see in \autoref{fig:black_white_roc} this indeed
significantly improved the feature vector. However it still seems best to use basic RGB decomposition.

\begin{figure}
    \centering
    \def\svgwidth{\columnwidth}
    \input{img/black_white_roc.pdf_tex}
    \caption[Effect of adding black and white bins to hue histograms]{Effect of adding black and white bins to hue histograms. The figure shows \gls{roc} curves of histogram based approaches of with RGB, hue \& saturation and only hue decomposition with the best performing number of bins in each category. The plot shows improved performance obtained by adding black and white bins to the latter two categories.}
    \label{fig:black_white_roc}
\end{figure}

\subsection{Evaluation of Deep Learning Approaches}

In the previous section we have focused on the feature vectors drawn using
histograms. Now, we explore the approaches involving \glspl{nn}. In this
section we make use of pre-trained model in Tensorflow framework
(\cite{tensorflow}) and described in \autoref{sec:existing_architectures}.

\subsubsection{Direct Use of Pre-trained Models}

Perhaps the simplest usage of pre-trained models is to use them directly
wihtout any additional training. The potential in such usage is that the
second to last layer (i.e. prior actual ``classification'' layer) has to
contain enough of information to classify the original image. Therefore,
we experiment whether such information is enough for our purposes.

As we already stated, we mainly use ResNet and MobileNet architectures
pre-trained on the ImageNet dataset. These archtecture, as is common,
requires the input of the fixed size. That means we need to rescale all the
input images. As the sizes of original bounding boxes are diverse (see
\autoref{fig:size_dist}, we can not avoid distorting the original images.
Therefore, we evaluate performance of the network with various input sizes.

\begin{figure}
    \centering
    \def\svgwidth{\columnwidth}
    \input{img/size_distribution.pdf_tex}
    \caption{Distribution of heights and widths of detections}
    \label{fig:size_dist}
\end{figure}

The \gls{roc} curves (with the same setting as in previous subsection) of the
annotation with the pre-trained models in \autoref{fig:pretrained_nn_roc}.
We have evaluated various input sizes. In case of MobileNet we have tested
all the input sizes that are available for the architecture. As for the 
ResNet we have tested lower number of smaller sizes, mainly for its higher
complexity and poorer performace compared to the MobileNet.

As we can see directly from the \gls{roc} curves, the decidedly best setting for this type of the \gls{nn} annotation is MobileNet with input resized to $128 \times 128$. 

\begin{figure}
    \centering
    \def\svgwidth{\columnwidth}
    \input{img/pretrained_nn_roc.pdf_tex}
    \caption{ROC curves of pre-trained model}
    \label{fig:pretrained_nn_roc}
\end{figure}

We have also experimented with different distance functions. As you can in \autoref{fig:basic_nn_roc_dist} the effect of different distance functions are rather small. Especially between Euclidean and Manhattan distance.

\begin{figure}
    \centering
    \def\svgwidth{\columnwidth}
    \input{img/basic_nn_roc_dist.pdf_tex}
    \caption{Comparison of effect of different distance functions on MobileNet with shape 128}
    \label{fig:basic_nn_roc_dist}
\end{figure}

\subsubsection{Fine-tuning the Pre-trained Models}

As we showed we achieved somewhat useful results with just using as-is pre-trained models. However such models were trained for classification tasks and therefore there should be a room for improvement when trained specifically on \reid{} task. For this reason we aim to train the networks (with weights initialized as trained on classification task) on our dataset.

For the first experiment in this regard we do not change the architecture at all. We just use the architecture as-is and train in on our dataset using Triplet Loss (\autoref{sec:triplet_loss}).

\begin{figure}
    \centering
    \def\svgwidth{\columnwidth}
    \large
    \scalebox{0.8}{\input{img/training_loss.pdf_tex}}
    \caption{Training loss during training}
    \label{fig:training_loss}
\end{figure}

As we can see from the log in \autoref{fig:training_loss} we significantly reduce the training error by this approach. However when we tried to apply the resulting \gls{nn} on our testing \gls{ses} (see \gls{roc} curve in \autoref{fig:basic_nn_overfit_roc}), we see that the performance of the network significantly decreased. Furthermore, we can see that the performance worsen even after the single epoch of training. The weights are being altered too much even in single epoch as the original capabilities of the network is lost.

\begin{figure}
    \centering
    \def\svgwidth{\columnwidth}
    \input{img/basic_nn_overfit_roc.pdf_tex}
    \caption{Performance of the network after training for several epochs}
    \label{fig:basic_nn_overfit_roc}
\end{figure}

We therefore lowered the learning rate to try to preserve the original information within the network. This proved to be the correct approach as we manage to significantly increase the performance of the network as you can see in \autoref{fig:basic_nn_lr1_roc} and \autoref{fig:basic_nn_lr2_roc}.

\begin{figure}
    \centering
    \def\svgwidth{\columnwidth}
    \input{img/basic_nn_lr1_roc.pdf_tex}
    \caption{Performance of the network after training with learning rate 0.0001}
    \label{fig:basic_nn_lr1_roc}
\end{figure}

\begin{figure}
    \centering
    \def\svgwidth{\columnwidth}
    \input{img/basic_nn_lr2_roc.pdf_tex}
    \caption{Performance of the network after training with learning rate 0.00002}
    \label{fig:basic_nn_lr2_roc}
\end{figure}

Now, we proceed evaluate effect of different distance functions. This also means to train a new \gls{nn} as used loss function (Triplet Loss -- \autoref{eq:triplet}) depends on the choice of the distance function. In other related work we see the Triplet Loss used either with Euclidean distance or Cosine distance (or alternatively squared Euclidean distance\footnote{As the final layer of our \glspl{nn} is normalizing layer, the square Euclidean distance and cosine distance are proportional to each other: $\delta_{euclid}(\vec{x}, \vec{y}) = 2 - 2\vec{x}^T\vec{y} = 2 - 2 \cos(\vec{x}, \vec{y})$ (as the norms of input vectors are 1).}). However, we also aim to verify usefulness of Manhattan distance, as it works well with the histogram approach.

As you can see in \todo{ref}, the with different distance functions (both cosine and Manhattan) we achieved better results. As we can see in detailed 


\chapter{Implementation}

\label{ch:implementation}

This chapter overviews the structure of the re-identification project implementation. For the detailed guide on how to use the attached software for \gls{iden} generation on a new dataset, please read \autoref{ch:guide}. For the implementation details, please refer to the attached code.

As we mentioned in the \autoref{sec:vl}, our whole work is designed as a module for \gls{VL}. Even though our work is largely independent of other modules within the framework of \gls{VL}, it shares one main component -- relation database. In this case, the database serves as an API between the modules and storage for the results.

In our case we worked with PostgreSQL as a management system. However, since we use SQLAlchemy library (\cite{sqlalchemy}) for all the communication with the database, we expect no compatibility issues with the usage of another management system.

\subsubsection*{Code Structure}
\todo[inline]{obrazok DB}
The implementation is split into several parts. Each part is independent of the others, using only the database as the common point (similarly to Videolytics modules). For the purpose of testing and evaluation, we also add a command-line interface for each part. With the database approach, we have two options on how to access the results. 
Alongside the parts we describe further in more detail,  we also provide several small scripts used for the evaluation in the attached code.

The first part -- feature vector part -- is responsible for creating feature vectors described in \autoref{ch:features}. The main pipeline for this module consists of downloading the data to compute the feature vectors from the database, annotate them with the feature vectors, and then upload the results back to the database. This process is used exactly as described for the histogram approaches. In the case of the neural networks, additionally, the data are downloaded for the training. In such a case, we first download the data, train, and save the network locally. Then we run the annotation process as described in the main pipeline with the trained network.

The second part -- clustering part -- is responsible for clustering the \glspl{det} (\autoref{ch:iden_construction}), that is for both -- trajectory generation and identity generation. For this process, the \glspl{det} from given cameras are again retrieved from the database along, if applicable, with the corresponding feature vectors created with the selected settings. If we use the construction of trajectories from before, they are also queried. Once the \glspl{det} are clustered according to the parameters, the resulting trajectories or \glspl{iden} are again committed to the database for use by other \gls{VL} modules. The options which were supplied for the \gls{iden} construction is also committed to the database as a new ``\gls{iden} model''. All the created \glspl{iden} are assigned to this model so they can be later queried if the same options are supplied.

\subsubsection*{Annotator}
We also provide a tool for manual annotation of the \glspl{det} we used to obtain the ground truth for our datasets. The tool queries the \glspl{det} from the database as well as selected \glspl{iden} or trajectory. User then can see the \glspl{det} frame-by-frame as well as their division to \glspl{iden}. It is possible to re-assign the wrongly assigned \glspl{det} to different \gls{iden}. This approach allows for semi-automated annotation: The \glspl{det} are firstly clustered by a (even poor) algorithm, then only wrongly \glspl{det} needs to be corrected. Once the re-assignment is complete, the resulting clustering can be committed back to the database as a new set of \glspl{iden} with a new ``identity model''.

\todo[inline]{screenshot z anotatoru}

\chapter*{Conclusion}
\addcontentsline{toc}{chapter}{Conclusion}

{\color{red} zhrnutie o com bola praca, aka bola contribution (ako to vylepsilo svet, ktory realny problem to riesi, atd) a ako sa to moze rozsitit (future work)
pripadne nejake remarks typu: "toto nam nevyslo ale nevadi lebo aspon sme overili ze to nejde"}


%%% Bibliography
\include{bibliography}

%%% Figures used in the thesis (consider if this is needed)
\listoffigures

%%% Tables used in the thesis (consider if this is needed)
%%% In mathematical theses, it could be better to move the list of tables to the beginning of the thesis.
\listoftables

%%% Abbreviations used in the thesis, if any, including their explanation
%%% In mathematical theses, it could be better to move the list of abbreviations to the beginning of the thesis.
%\chapwithtoc{List of Abbreviations}

\clearpage
\printnoidxglossaries

%%% Attachments to the master thesis, if any. Each attachment must be
%%% referred to at least once from the text of the thesis. Attachments
%%% are numbered.
%%%
%%% The printed version should preferably contain attachments, which can be
%%% read (additional tables and charts, supplementary text, examples of
%%% program output, etc.). The electronic version is more suited for attachments
%%% which will likely be used in an electronic form rather than read (program
%%% source code, data files, interactive charts, etc.). Electronic attachments
%%% should be uploaded to SIS and optionally also included in the thesis on a~CD/DVD.
%%% Allowed file formats are specified in provision of the rector no. 72/2017.
\appendix
\chapter{User Guide}

\label{ch:guide}
%\chapter{Attachments}
%
%\section{First Attachment}

\clearpage
\listoftodos

\openright
\end{document}
